%% Generated by Sphinx.
\def\sphinxdocclass{report}
\documentclass[letterpaper,10pt,english]{sphinxmanual}
\ifdefined\pdfpxdimen
   \let\sphinxpxdimen\pdfpxdimen\else\newdimen\sphinxpxdimen
\fi \sphinxpxdimen=.75bp\relax
\ifdefined\pdfimageresolution
    \pdfimageresolution= \numexpr \dimexpr1in\relax/\sphinxpxdimen\relax
\fi
%% let collapsible pdf bookmarks panel have high depth per default
\PassOptionsToPackage{bookmarksdepth=5}{hyperref}

\PassOptionsToPackage{booktabs}{sphinx}
\PassOptionsToPackage{colorrows}{sphinx}

\PassOptionsToPackage{warn}{textcomp}
\usepackage[utf8]{inputenc}
\ifdefined\DeclareUnicodeCharacter
% support both utf8 and utf8x syntaxes
  \ifdefined\DeclareUnicodeCharacterAsOptional
    \def\sphinxDUC#1{\DeclareUnicodeCharacter{"#1}}
  \else
    \let\sphinxDUC\DeclareUnicodeCharacter
  \fi
  \sphinxDUC{00A0}{\nobreakspace}
  \sphinxDUC{2500}{\sphinxunichar{2500}}
  \sphinxDUC{2502}{\sphinxunichar{2502}}
  \sphinxDUC{2514}{\sphinxunichar{2514}}
  \sphinxDUC{251C}{\sphinxunichar{251C}}
  \sphinxDUC{2572}{\textbackslash}
\fi
\usepackage{cmap}
\usepackage[T1]{fontenc}
\usepackage{amsmath,amssymb,amstext}
\usepackage{babel}



\usepackage{tgtermes}
\usepackage{tgheros}
\renewcommand{\ttdefault}{txtt}



\usepackage[Bjarne]{fncychap}
\usepackage{sphinx}

\fvset{fontsize=auto}
\usepackage{geometry}


% Include hyperref last.
\usepackage{hyperref}
% Fix anchor placement for figures with captions.
\usepackage{hypcap}% it must be loaded after hyperref.
% Set up styles of URL: it should be placed after hyperref.
\urlstyle{same}

\addto\captionsenglish{\renewcommand{\contentsname}{Contents:}}

\usepackage{sphinxmessages}
\setcounter{tocdepth}{1}



\title{pysewer}
\date{Oct 20, 2023}
\release{0.1.13}
\author{UFZ}
\newcommand{\sphinxlogo}{\vbox{}}
\renewcommand{\releasename}{Release}
\makeindex
\begin{document}

\ifdefined\shorthandoff
  \ifnum\catcode`\=\string=\active\shorthandoff{=}\fi
  \ifnum\catcode`\"=\active\shorthandoff{"}\fi
\fi

\pagestyle{empty}
\sphinxmaketitle
\pagestyle{plain}
\sphinxtableofcontents
\pagestyle{normal}
\phantomsection\label{\detokenize{index::doc}}


\sphinxstepscope


\chapter{About}
\label{\detokenize{about:about}}\label{\detokenize{about::doc}}
\sphinxAtStartPar
Pysewer provides a framework to automatically generate cost\sphinxhyphen{}efficient sewer network layouts using minimal data requirements.

\sphinxAtStartPar
It is build around an algorithm for generation of viable sewer\sphinxhyphen{}network layouts. The approximated sewer network is represented by sources (households/buildings), potential pathways, and one or multiple sinks.
The algorithm approximates the directed Steiner tree (the Steiner arborescence) between all sources and the sink by using an repeated shortest path heuristic (RSPH).


\section{Key features}
\label{\detokenize{about:key-features}}\begin{itemize}
\item {} 
\sphinxAtStartPar
Automatic generation of gravity\sphinxhyphen{}flow prioritised sewer network layouts

\item {} 
\sphinxAtStartPar
Minimal data requirements

\item {} 
\sphinxAtStartPar
Sewer network visualisation

\item {} 
\sphinxAtStartPar
Export of the sewer network for preliminary planning and analysis

\end{itemize}


\section{Who is it for?}
\label{\detokenize{about:who-is-it-for}}
\sphinxAtStartPar
Pysewer is for anyone who wants to generate a sewer network layout with minimal data requirements.
It is designed for sanitary engineers, critical infrastructure planners, researchers who need to generate sewer network layouts for preliminary planning and analysis.


\section{Development team}
\label{\detokenize{about:development-team}}
\sphinxAtStartPar
Pysewer is developed by the working group on Water\sphinxhyphen{}Sensitive Infrastructure Planning (WASP) at the Helmholtz Centre for Environmental Research (UFZ) in Leipzig, Germany.

\sphinxstepscope

\sphinxAtStartPar
\# SPDX\sphinxhyphen{}FileCopyrightText: 2023 Helmholtz Centre for Environmental Research (UFZ)
\# SPDX\sphinxhyphen{}License\sphinxhyphen{}Identifier: GPL\sphinxhyphen{}3.0\sphinxhyphen{}only


\chapter{Installation}
\label{\detokenize{install:installation}}\label{\detokenize{install::doc}}
\sphinxAtStartPar
Currently the installation is easiest managed via Anaconda. Anaconda 3 can be downloaded \sphinxhref{https://www.anaconda.com/products/individual}{here}


\section{Create a new environment}
\label{\detokenize{install:create-a-new-environment}}
\sphinxAtStartPar
First, we want to create a new environment in Anaconda. Therefore, we open Anaconda prompt and create a new Python 3.10 Environment and name it pysewer by running the following command:

\begin{sphinxVerbatim}[commandchars=\\\{\}]
\PYG{n}{conda} \PYG{n}{create} \PYG{o}{\PYGZhy{}}\PYG{n}{n} \PYG{n}{pysewer} \PYG{n}{python}\PYG{o}{=}\PYG{l+m+mf}{3.10}
\end{sphinxVerbatim}

\sphinxAtStartPar
We can then activate the environment by running:

\begin{sphinxVerbatim}[commandchars=\\\{\}]
\PYG{n}{conda} \PYG{n}{activate} \PYG{n}{pysewer}
\end{sphinxVerbatim}

\sphinxAtStartPar
For installing GDAL, rasterio and fiona :

\begin{sphinxVerbatim}[commandchars=\\\{\}]
\PYG{n}{conda} \PYG{n}{install} \PYG{o}{\PYGZhy{}}\PYG{n}{c} \PYG{n}{conda}\PYG{o}{\PYGZhy{}}\PYG{n}{forge} \PYG{n}{gdal} \PYG{n}{rasterio} \PYG{n}{fiona}
\end{sphinxVerbatim}


\section{Install pysewer via pip}
\label{\detokenize{install:install-pysewer-via-pip}}
\sphinxAtStartPar
You can either get pysewer and install it using git and pip with:

\begin{sphinxVerbatim}[commandchars=\\\{\}]
git\PYG{+w}{ }clone\PYG{+w}{ }https://git.ufz.de/despot/pysewer\PYGZus{}dev.git
\PYG{n+nb}{cd}\PYG{+w}{ }pysewer
pip\PYG{+w}{ }install\PYG{+w}{ }.
\PYG{c+c1}{\PYGZsh{} for the development version}
python\PYG{+w}{ }\PYGZhy{}m\PYG{+w}{ }pip\PYG{+w}{ }install\PYG{+w}{ }\PYGZhy{}e\PYG{+w}{ }.
\end{sphinxVerbatim}

\sphinxstepscope


\chapter{Pysewer Modules}
\label{\detokenize{modules:pysewer-modules}}\label{\detokenize{modules::doc}}
\sphinxstepscope


\section{Default settings}
\label{\detokenize{pysewer:module-pysewer.config.settings}}\label{\detokenize{pysewer:default-settings}}\label{\detokenize{pysewer::doc}}\index{module@\spxentry{module}!pysewer.config.settings@\spxentry{pysewer.config.settings}}\index{pysewer.config.settings@\spxentry{pysewer.config.settings}!module@\spxentry{module}}\index{Config (class in pysewer.config.settings)@\spxentry{Config}\spxextra{class in pysewer.config.settings}}

\begin{fulllineitems}
\phantomsection\label{\detokenize{pysewer:pysewer.config.settings.Config}}
\pysigstartsignatures
\pysiglinewithargsret{\sphinxbfcode{\sphinxupquote{class\DUrole{w}{ }}}\sphinxcode{\sphinxupquote{pysewer.config.settings.}}\sphinxbfcode{\sphinxupquote{Config}}}{\sphinxparam{\DUrole{n}{preprocessing}\DUrole{p}{:}\DUrole{w}{ }\DUrole{n}{pysewer.config.settings.Preporocessing}}\sphinxparamcomma \sphinxparam{\DUrole{n}{optimization}\DUrole{p}{:}\DUrole{w}{ }\DUrole{n}{pysewer.config.settings.Optimization}}\sphinxparamcomma \sphinxparam{\DUrole{n}{plotting}\DUrole{p}{:}\DUrole{w}{ }\DUrole{n}{pysewer.config.settings.Plotting}}\sphinxparamcomma \sphinxparam{\DUrole{n}{export}\DUrole{p}{:}\DUrole{w}{ }\DUrole{n}{pysewer.config.settings.Export}}}{}
\pysigstopsignatures
\sphinxAtStartPar
Bases: \sphinxcode{\sphinxupquote{object}}
\index{export (pysewer.config.settings.Config attribute)@\spxentry{export}\spxextra{pysewer.config.settings.Config attribute}}

\begin{fulllineitems}
\phantomsection\label{\detokenize{pysewer:pysewer.config.settings.Config.export}}
\pysigstartsignatures
\pysigline{\sphinxbfcode{\sphinxupquote{export}}\sphinxbfcode{\sphinxupquote{\DUrole{p}{:}\DUrole{w}{ }Export}}}
\pysigstopsignatures
\end{fulllineitems}

\index{optimization (pysewer.config.settings.Config attribute)@\spxentry{optimization}\spxextra{pysewer.config.settings.Config attribute}}

\begin{fulllineitems}
\phantomsection\label{\detokenize{pysewer:pysewer.config.settings.Config.optimization}}
\pysigstartsignatures
\pysigline{\sphinxbfcode{\sphinxupquote{optimization}}\sphinxbfcode{\sphinxupquote{\DUrole{p}{:}\DUrole{w}{ }Optimization}}}
\pysigstopsignatures
\end{fulllineitems}

\index{plotting (pysewer.config.settings.Config attribute)@\spxentry{plotting}\spxextra{pysewer.config.settings.Config attribute}}

\begin{fulllineitems}
\phantomsection\label{\detokenize{pysewer:pysewer.config.settings.Config.plotting}}
\pysigstartsignatures
\pysigline{\sphinxbfcode{\sphinxupquote{plotting}}\sphinxbfcode{\sphinxupquote{\DUrole{p}{:}\DUrole{w}{ }Plotting}}}
\pysigstopsignatures
\end{fulllineitems}

\index{preprocessing (pysewer.config.settings.Config attribute)@\spxentry{preprocessing}\spxextra{pysewer.config.settings.Config attribute}}

\begin{fulllineitems}
\phantomsection\label{\detokenize{pysewer:pysewer.config.settings.Config.preprocessing}}
\pysigstartsignatures
\pysigline{\sphinxbfcode{\sphinxupquote{preprocessing}}\sphinxbfcode{\sphinxupquote{\DUrole{p}{:}\DUrole{w}{ }Preporocessing}}}
\pysigstopsignatures
\end{fulllineitems}


\end{fulllineitems}

\index{Export (class in pysewer.config.settings)@\spxentry{Export}\spxextra{class in pysewer.config.settings}}

\begin{fulllineitems}
\phantomsection\label{\detokenize{pysewer:pysewer.config.settings.Export}}
\pysigstartsignatures
\pysiglinewithargsret{\sphinxbfcode{\sphinxupquote{class\DUrole{w}{ }}}\sphinxcode{\sphinxupquote{pysewer.config.settings.}}\sphinxbfcode{\sphinxupquote{Export}}}{\sphinxparam{\DUrole{n}{file\_format}\DUrole{p}{:}\DUrole{w}{ }\DUrole{n}{str}}}{}
\pysigstopsignatures
\sphinxAtStartPar
Bases: \sphinxcode{\sphinxupquote{object}}
\index{file\_format (pysewer.config.settings.Export attribute)@\spxentry{file\_format}\spxextra{pysewer.config.settings.Export attribute}}

\begin{fulllineitems}
\phantomsection\label{\detokenize{pysewer:pysewer.config.settings.Export.file_format}}
\pysigstartsignatures
\pysigline{\sphinxbfcode{\sphinxupquote{file\_format}}\sphinxbfcode{\sphinxupquote{\DUrole{p}{:}\DUrole{w}{ }str}}}
\pysigstopsignatures
\end{fulllineitems}


\end{fulllineitems}

\index{Optimization (class in pysewer.config.settings)@\spxentry{Optimization}\spxextra{class in pysewer.config.settings}}

\begin{fulllineitems}
\phantomsection\label{\detokenize{pysewer:pysewer.config.settings.Optimization}}
\pysigstartsignatures
\pysiglinewithargsret{\sphinxbfcode{\sphinxupquote{class\DUrole{w}{ }}}\sphinxcode{\sphinxupquote{pysewer.config.settings.}}\sphinxbfcode{\sphinxupquote{Optimization}}}{\sphinxparam{\DUrole{n}{inhabitants\_dwelling}\DUrole{p}{:}\DUrole{w}{ }\DUrole{n}{int}}\sphinxparamcomma \sphinxparam{\DUrole{n}{daily\_wastewater\_person}\DUrole{p}{:}\DUrole{w}{ }\DUrole{n}{float}}\sphinxparamcomma \sphinxparam{\DUrole{n}{peak\_factor}\DUrole{p}{:}\DUrole{w}{ }\DUrole{n}{float}}\sphinxparamcomma \sphinxparam{\DUrole{n}{min\_slope}\DUrole{p}{:}\DUrole{w}{ }\DUrole{n}{float}}\sphinxparamcomma \sphinxparam{\DUrole{n}{tmax}\DUrole{p}{:}\DUrole{w}{ }\DUrole{n}{float}}\sphinxparamcomma \sphinxparam{\DUrole{n}{tmin}\DUrole{p}{:}\DUrole{w}{ }\DUrole{n}{float}}\sphinxparamcomma \sphinxparam{\DUrole{n}{inflow\_trench\_depth}\DUrole{p}{:}\DUrole{w}{ }\DUrole{n}{float}}\sphinxparamcomma \sphinxparam{\DUrole{n}{min\_trench\_depth}\DUrole{p}{:}\DUrole{w}{ }\DUrole{n}{float}}\sphinxparamcomma \sphinxparam{\DUrole{n}{diameters}\DUrole{p}{:}\DUrole{w}{ }\DUrole{n}{List\DUrole{p}{{[}}float\DUrole{p}{{]}}}}\sphinxparamcomma \sphinxparam{\DUrole{n}{roughness}\DUrole{p}{:}\DUrole{w}{ }\DUrole{n}{float}}\sphinxparamcomma \sphinxparam{\DUrole{n}{pressurized\_diameter}\DUrole{p}{:}\DUrole{w}{ }\DUrole{n}{float}}}{}
\pysigstopsignatures
\sphinxAtStartPar
Bases: \sphinxcode{\sphinxupquote{object}}
\index{daily\_wastewater\_person (pysewer.config.settings.Optimization attribute)@\spxentry{daily\_wastewater\_person}\spxextra{pysewer.config.settings.Optimization attribute}}

\begin{fulllineitems}
\phantomsection\label{\detokenize{pysewer:pysewer.config.settings.Optimization.daily_wastewater_person}}
\pysigstartsignatures
\pysigline{\sphinxbfcode{\sphinxupquote{daily\_wastewater\_person}}\sphinxbfcode{\sphinxupquote{\DUrole{p}{:}\DUrole{w}{ }float}}}
\pysigstopsignatures
\end{fulllineitems}

\index{diameters (pysewer.config.settings.Optimization attribute)@\spxentry{diameters}\spxextra{pysewer.config.settings.Optimization attribute}}

\begin{fulllineitems}
\phantomsection\label{\detokenize{pysewer:pysewer.config.settings.Optimization.diameters}}
\pysigstartsignatures
\pysigline{\sphinxbfcode{\sphinxupquote{diameters}}\sphinxbfcode{\sphinxupquote{\DUrole{p}{:}\DUrole{w}{ }List\DUrole{p}{{[}}float\DUrole{p}{{]}}}}}
\pysigstopsignatures
\end{fulllineitems}

\index{inflow\_trench\_depth (pysewer.config.settings.Optimization attribute)@\spxentry{inflow\_trench\_depth}\spxextra{pysewer.config.settings.Optimization attribute}}

\begin{fulllineitems}
\phantomsection\label{\detokenize{pysewer:pysewer.config.settings.Optimization.inflow_trench_depth}}
\pysigstartsignatures
\pysigline{\sphinxbfcode{\sphinxupquote{inflow\_trench\_depth}}\sphinxbfcode{\sphinxupquote{\DUrole{p}{:}\DUrole{w}{ }float}}}
\pysigstopsignatures
\end{fulllineitems}

\index{inhabitants\_dwelling (pysewer.config.settings.Optimization attribute)@\spxentry{inhabitants\_dwelling}\spxextra{pysewer.config.settings.Optimization attribute}}

\begin{fulllineitems}
\phantomsection\label{\detokenize{pysewer:pysewer.config.settings.Optimization.inhabitants_dwelling}}
\pysigstartsignatures
\pysigline{\sphinxbfcode{\sphinxupquote{inhabitants\_dwelling}}\sphinxbfcode{\sphinxupquote{\DUrole{p}{:}\DUrole{w}{ }int}}}
\pysigstopsignatures
\end{fulllineitems}

\index{min\_slope (pysewer.config.settings.Optimization attribute)@\spxentry{min\_slope}\spxextra{pysewer.config.settings.Optimization attribute}}

\begin{fulllineitems}
\phantomsection\label{\detokenize{pysewer:pysewer.config.settings.Optimization.min_slope}}
\pysigstartsignatures
\pysigline{\sphinxbfcode{\sphinxupquote{min\_slope}}\sphinxbfcode{\sphinxupquote{\DUrole{p}{:}\DUrole{w}{ }float}}}
\pysigstopsignatures
\end{fulllineitems}

\index{min\_trench\_depth (pysewer.config.settings.Optimization attribute)@\spxentry{min\_trench\_depth}\spxextra{pysewer.config.settings.Optimization attribute}}

\begin{fulllineitems}
\phantomsection\label{\detokenize{pysewer:pysewer.config.settings.Optimization.min_trench_depth}}
\pysigstartsignatures
\pysigline{\sphinxbfcode{\sphinxupquote{min\_trench\_depth}}\sphinxbfcode{\sphinxupquote{\DUrole{p}{:}\DUrole{w}{ }float}}}
\pysigstopsignatures
\end{fulllineitems}

\index{peak\_factor (pysewer.config.settings.Optimization attribute)@\spxentry{peak\_factor}\spxextra{pysewer.config.settings.Optimization attribute}}

\begin{fulllineitems}
\phantomsection\label{\detokenize{pysewer:pysewer.config.settings.Optimization.peak_factor}}
\pysigstartsignatures
\pysigline{\sphinxbfcode{\sphinxupquote{peak\_factor}}\sphinxbfcode{\sphinxupquote{\DUrole{p}{:}\DUrole{w}{ }float}}}
\pysigstopsignatures
\end{fulllineitems}

\index{pressurized\_diameter (pysewer.config.settings.Optimization attribute)@\spxentry{pressurized\_diameter}\spxextra{pysewer.config.settings.Optimization attribute}}

\begin{fulllineitems}
\phantomsection\label{\detokenize{pysewer:pysewer.config.settings.Optimization.pressurized_diameter}}
\pysigstartsignatures
\pysigline{\sphinxbfcode{\sphinxupquote{pressurized\_diameter}}\sphinxbfcode{\sphinxupquote{\DUrole{p}{:}\DUrole{w}{ }float}}}
\pysigstopsignatures
\end{fulllineitems}

\index{roughness (pysewer.config.settings.Optimization attribute)@\spxentry{roughness}\spxextra{pysewer.config.settings.Optimization attribute}}

\begin{fulllineitems}
\phantomsection\label{\detokenize{pysewer:pysewer.config.settings.Optimization.roughness}}
\pysigstartsignatures
\pysigline{\sphinxbfcode{\sphinxupquote{roughness}}\sphinxbfcode{\sphinxupquote{\DUrole{p}{:}\DUrole{w}{ }float}}}
\pysigstopsignatures
\end{fulllineitems}

\index{tmax (pysewer.config.settings.Optimization attribute)@\spxentry{tmax}\spxextra{pysewer.config.settings.Optimization attribute}}

\begin{fulllineitems}
\phantomsection\label{\detokenize{pysewer:pysewer.config.settings.Optimization.tmax}}
\pysigstartsignatures
\pysigline{\sphinxbfcode{\sphinxupquote{tmax}}\sphinxbfcode{\sphinxupquote{\DUrole{p}{:}\DUrole{w}{ }float}}}
\pysigstopsignatures
\end{fulllineitems}

\index{tmin (pysewer.config.settings.Optimization attribute)@\spxentry{tmin}\spxextra{pysewer.config.settings.Optimization attribute}}

\begin{fulllineitems}
\phantomsection\label{\detokenize{pysewer:pysewer.config.settings.Optimization.tmin}}
\pysigstartsignatures
\pysigline{\sphinxbfcode{\sphinxupquote{tmin}}\sphinxbfcode{\sphinxupquote{\DUrole{p}{:}\DUrole{w}{ }float}}}
\pysigstopsignatures
\end{fulllineitems}


\end{fulllineitems}

\index{Plotting (class in pysewer.config.settings)@\spxentry{Plotting}\spxextra{class in pysewer.config.settings}}

\begin{fulllineitems}
\phantomsection\label{\detokenize{pysewer:pysewer.config.settings.Plotting}}
\pysigstartsignatures
\pysiglinewithargsret{\sphinxbfcode{\sphinxupquote{class\DUrole{w}{ }}}\sphinxcode{\sphinxupquote{pysewer.config.settings.}}\sphinxbfcode{\sphinxupquote{Plotting}}}{\sphinxparam{\DUrole{n}{plot\_connection\_graph}\DUrole{p}{:}\DUrole{w}{ }\DUrole{n}{bool}}\sphinxparamcomma \sphinxparam{\DUrole{n}{plot\_junction\_graph}\DUrole{p}{:}\DUrole{w}{ }\DUrole{n}{bool}}\sphinxparamcomma \sphinxparam{\DUrole{n}{plot\_sink}\DUrole{p}{:}\DUrole{w}{ }\DUrole{n}{bool}}\sphinxparamcomma \sphinxparam{\DUrole{n}{plot\_sewer}\DUrole{p}{:}\DUrole{w}{ }\DUrole{n}{bool}}\sphinxparamcomma \sphinxparam{\DUrole{n}{hillshade}\DUrole{p}{:}\DUrole{w}{ }\DUrole{n}{bool}}\sphinxparamcomma \sphinxparam{\DUrole{n}{colormap}\DUrole{p}{:}\DUrole{w}{ }\DUrole{n}{str}}\sphinxparamcomma \sphinxparam{\DUrole{n}{sewer\_graph}\DUrole{p}{:}\DUrole{w}{ }\DUrole{n}{networkx.classes.graph.Graph}\DUrole{w}{ }\DUrole{o}{=}\DUrole{w}{ }\DUrole{default_value}{None}}\sphinxparamcomma \sphinxparam{\DUrole{n}{info\_table}\DUrole{p}{:}\DUrole{w}{ }\DUrole{n}{dict\DUrole{w}{ }\DUrole{p}{|}\DUrole{w}{ }None}\DUrole{w}{ }\DUrole{o}{=}\DUrole{w}{ }\DUrole{default_value}{None}}}{}
\pysigstopsignatures
\sphinxAtStartPar
Bases: \sphinxcode{\sphinxupquote{object}}
\index{colormap (pysewer.config.settings.Plotting attribute)@\spxentry{colormap}\spxextra{pysewer.config.settings.Plotting attribute}}

\begin{fulllineitems}
\phantomsection\label{\detokenize{pysewer:pysewer.config.settings.Plotting.colormap}}
\pysigstartsignatures
\pysigline{\sphinxbfcode{\sphinxupquote{colormap}}\sphinxbfcode{\sphinxupquote{\DUrole{p}{:}\DUrole{w}{ }str}}}
\pysigstopsignatures
\end{fulllineitems}

\index{hillshade (pysewer.config.settings.Plotting attribute)@\spxentry{hillshade}\spxextra{pysewer.config.settings.Plotting attribute}}

\begin{fulllineitems}
\phantomsection\label{\detokenize{pysewer:pysewer.config.settings.Plotting.hillshade}}
\pysigstartsignatures
\pysigline{\sphinxbfcode{\sphinxupquote{hillshade}}\sphinxbfcode{\sphinxupquote{\DUrole{p}{:}\DUrole{w}{ }bool}}}
\pysigstopsignatures
\end{fulllineitems}

\index{info\_table (pysewer.config.settings.Plotting attribute)@\spxentry{info\_table}\spxextra{pysewer.config.settings.Plotting attribute}}

\begin{fulllineitems}
\phantomsection\label{\detokenize{pysewer:pysewer.config.settings.Plotting.info_table}}
\pysigstartsignatures
\pysigline{\sphinxbfcode{\sphinxupquote{info\_table}}\sphinxbfcode{\sphinxupquote{\DUrole{p}{:}\DUrole{w}{ }dict\DUrole{w}{ }\DUrole{p}{|}\DUrole{w}{ }None}}\sphinxbfcode{\sphinxupquote{\DUrole{w}{ }\DUrole{p}{=}\DUrole{w}{ }None}}}
\pysigstopsignatures
\end{fulllineitems}

\index{plot\_connection\_graph (pysewer.config.settings.Plotting attribute)@\spxentry{plot\_connection\_graph}\spxextra{pysewer.config.settings.Plotting attribute}}

\begin{fulllineitems}
\phantomsection\label{\detokenize{pysewer:pysewer.config.settings.Plotting.plot_connection_graph}}
\pysigstartsignatures
\pysigline{\sphinxbfcode{\sphinxupquote{plot\_connection\_graph}}\sphinxbfcode{\sphinxupquote{\DUrole{p}{:}\DUrole{w}{ }bool}}}
\pysigstopsignatures
\end{fulllineitems}

\index{plot\_junction\_graph (pysewer.config.settings.Plotting attribute)@\spxentry{plot\_junction\_graph}\spxextra{pysewer.config.settings.Plotting attribute}}

\begin{fulllineitems}
\phantomsection\label{\detokenize{pysewer:pysewer.config.settings.Plotting.plot_junction_graph}}
\pysigstartsignatures
\pysigline{\sphinxbfcode{\sphinxupquote{plot\_junction\_graph}}\sphinxbfcode{\sphinxupquote{\DUrole{p}{:}\DUrole{w}{ }bool}}}
\pysigstopsignatures
\end{fulllineitems}

\index{plot\_sewer (pysewer.config.settings.Plotting attribute)@\spxentry{plot\_sewer}\spxextra{pysewer.config.settings.Plotting attribute}}

\begin{fulllineitems}
\phantomsection\label{\detokenize{pysewer:pysewer.config.settings.Plotting.plot_sewer}}
\pysigstartsignatures
\pysigline{\sphinxbfcode{\sphinxupquote{plot\_sewer}}\sphinxbfcode{\sphinxupquote{\DUrole{p}{:}\DUrole{w}{ }bool}}}
\pysigstopsignatures
\end{fulllineitems}

\index{plot\_sink (pysewer.config.settings.Plotting attribute)@\spxentry{plot\_sink}\spxextra{pysewer.config.settings.Plotting attribute}}

\begin{fulllineitems}
\phantomsection\label{\detokenize{pysewer:pysewer.config.settings.Plotting.plot_sink}}
\pysigstartsignatures
\pysigline{\sphinxbfcode{\sphinxupquote{plot\_sink}}\sphinxbfcode{\sphinxupquote{\DUrole{p}{:}\DUrole{w}{ }bool}}}
\pysigstopsignatures
\end{fulllineitems}

\index{sewer\_graph (pysewer.config.settings.Plotting attribute)@\spxentry{sewer\_graph}\spxextra{pysewer.config.settings.Plotting attribute}}

\begin{fulllineitems}
\phantomsection\label{\detokenize{pysewer:pysewer.config.settings.Plotting.sewer_graph}}
\pysigstartsignatures
\pysigline{\sphinxbfcode{\sphinxupquote{sewer\_graph}}\sphinxbfcode{\sphinxupquote{\DUrole{p}{:}\DUrole{w}{ }Graph}}\sphinxbfcode{\sphinxupquote{\DUrole{w}{ }\DUrole{p}{=}\DUrole{w}{ }None}}}
\pysigstopsignatures
\end{fulllineitems}


\end{fulllineitems}

\index{Preporocessing (class in pysewer.config.settings)@\spxentry{Preporocessing}\spxextra{class in pysewer.config.settings}}

\begin{fulllineitems}
\phantomsection\label{\detokenize{pysewer:pysewer.config.settings.Preporocessing}}
\pysigstartsignatures
\pysiglinewithargsret{\sphinxbfcode{\sphinxupquote{class\DUrole{w}{ }}}\sphinxcode{\sphinxupquote{pysewer.config.settings.}}\sphinxbfcode{\sphinxupquote{Preporocessing}}}{\sphinxparam{\DUrole{n}{dem\_file\_path}\DUrole{p}{:}\DUrole{w}{ }\DUrole{n}{str\DUrole{w}{ }\DUrole{p}{|}\DUrole{w}{ }None}}\sphinxparamcomma \sphinxparam{\DUrole{n}{roads\_input\_data}\DUrole{p}{:}\DUrole{w}{ }\DUrole{n}{str\DUrole{w}{ }\DUrole{p}{|}\DUrole{w}{ }geopandas.geodataframe.GeoDataFrame\DUrole{w}{ }\DUrole{p}{|}\DUrole{w}{ }NoneType}}\sphinxparamcomma \sphinxparam{\DUrole{n}{buildings\_input\_data}\DUrole{p}{:}\DUrole{w}{ }\DUrole{n}{str\DUrole{w}{ }\DUrole{p}{|}\DUrole{w}{ }geopandas.geodataframe.GeoDataFrame\DUrole{w}{ }\DUrole{p}{|}\DUrole{w}{ }NoneType}}\sphinxparamcomma \sphinxparam{\DUrole{n}{dx}\DUrole{p}{:}\DUrole{w}{ }\DUrole{n}{int}}\sphinxparamcomma \sphinxparam{\DUrole{n}{pump\_penalty}\DUrole{p}{:}\DUrole{w}{ }\DUrole{n}{int}}\sphinxparamcomma \sphinxparam{\DUrole{n}{max\_connection\_length}\DUrole{p}{:}\DUrole{w}{ }\DUrole{n}{int}}\sphinxparamcomma \sphinxparam{\DUrole{n}{clustering}\DUrole{p}{:}\DUrole{w}{ }\DUrole{n}{str}}\sphinxparamcomma \sphinxparam{\DUrole{n}{connect\_buildings}\DUrole{p}{:}\DUrole{w}{ }\DUrole{n}{bool}}\sphinxparamcomma \sphinxparam{\DUrole{n}{add\_private\_sewer}\DUrole{p}{:}\DUrole{w}{ }\DUrole{n}{bool}}\sphinxparamcomma \sphinxparam{\DUrole{n}{field\_get\_sinks}\DUrole{p}{:}\DUrole{w}{ }\DUrole{n}{str}}\sphinxparamcomma \sphinxparam{\DUrole{n}{field\_get\_buildings}\DUrole{p}{:}\DUrole{w}{ }\DUrole{n}{str}}\sphinxparamcomma \sphinxparam{\DUrole{n}{value\_get\_sinks}\DUrole{p}{:}\DUrole{w}{ }\DUrole{n}{str}}\sphinxparamcomma \sphinxparam{\DUrole{n}{value\_get\_buildings}\DUrole{p}{:}\DUrole{w}{ }\DUrole{n}{str}}}{}
\pysigstopsignatures
\sphinxAtStartPar
Bases: \sphinxcode{\sphinxupquote{object}}
\index{add\_private\_sewer (pysewer.config.settings.Preporocessing attribute)@\spxentry{add\_private\_sewer}\spxextra{pysewer.config.settings.Preporocessing attribute}}

\begin{fulllineitems}
\phantomsection\label{\detokenize{pysewer:pysewer.config.settings.Preporocessing.add_private_sewer}}
\pysigstartsignatures
\pysigline{\sphinxbfcode{\sphinxupquote{add\_private\_sewer}}\sphinxbfcode{\sphinxupquote{\DUrole{p}{:}\DUrole{w}{ }bool}}}
\pysigstopsignatures
\end{fulllineitems}

\index{buildings\_input\_data (pysewer.config.settings.Preporocessing attribute)@\spxentry{buildings\_input\_data}\spxextra{pysewer.config.settings.Preporocessing attribute}}

\begin{fulllineitems}
\phantomsection\label{\detokenize{pysewer:pysewer.config.settings.Preporocessing.buildings_input_data}}
\pysigstartsignatures
\pysigline{\sphinxbfcode{\sphinxupquote{buildings\_input\_data}}\sphinxbfcode{\sphinxupquote{\DUrole{p}{:}\DUrole{w}{ }str\DUrole{w}{ }\DUrole{p}{|}\DUrole{w}{ }GeoDataFrame\DUrole{w}{ }\DUrole{p}{|}\DUrole{w}{ }None}}}
\pysigstopsignatures
\end{fulllineitems}

\index{clustering (pysewer.config.settings.Preporocessing attribute)@\spxentry{clustering}\spxextra{pysewer.config.settings.Preporocessing attribute}}

\begin{fulllineitems}
\phantomsection\label{\detokenize{pysewer:pysewer.config.settings.Preporocessing.clustering}}
\pysigstartsignatures
\pysigline{\sphinxbfcode{\sphinxupquote{clustering}}\sphinxbfcode{\sphinxupquote{\DUrole{p}{:}\DUrole{w}{ }str}}}
\pysigstopsignatures
\end{fulllineitems}

\index{connect\_buildings (pysewer.config.settings.Preporocessing attribute)@\spxentry{connect\_buildings}\spxextra{pysewer.config.settings.Preporocessing attribute}}

\begin{fulllineitems}
\phantomsection\label{\detokenize{pysewer:pysewer.config.settings.Preporocessing.connect_buildings}}
\pysigstartsignatures
\pysigline{\sphinxbfcode{\sphinxupquote{connect\_buildings}}\sphinxbfcode{\sphinxupquote{\DUrole{p}{:}\DUrole{w}{ }bool}}}
\pysigstopsignatures
\end{fulllineitems}

\index{dem\_file\_path (pysewer.config.settings.Preporocessing attribute)@\spxentry{dem\_file\_path}\spxextra{pysewer.config.settings.Preporocessing attribute}}

\begin{fulllineitems}
\phantomsection\label{\detokenize{pysewer:pysewer.config.settings.Preporocessing.dem_file_path}}
\pysigstartsignatures
\pysigline{\sphinxbfcode{\sphinxupquote{dem\_file\_path}}\sphinxbfcode{\sphinxupquote{\DUrole{p}{:}\DUrole{w}{ }str\DUrole{w}{ }\DUrole{p}{|}\DUrole{w}{ }None}}}
\pysigstopsignatures
\end{fulllineitems}

\index{dx (pysewer.config.settings.Preporocessing attribute)@\spxentry{dx}\spxextra{pysewer.config.settings.Preporocessing attribute}}

\begin{fulllineitems}
\phantomsection\label{\detokenize{pysewer:pysewer.config.settings.Preporocessing.dx}}
\pysigstartsignatures
\pysigline{\sphinxbfcode{\sphinxupquote{dx}}\sphinxbfcode{\sphinxupquote{\DUrole{p}{:}\DUrole{w}{ }int}}}
\pysigstopsignatures
\end{fulllineitems}

\index{field\_get\_buildings (pysewer.config.settings.Preporocessing attribute)@\spxentry{field\_get\_buildings}\spxextra{pysewer.config.settings.Preporocessing attribute}}

\begin{fulllineitems}
\phantomsection\label{\detokenize{pysewer:pysewer.config.settings.Preporocessing.field_get_buildings}}
\pysigstartsignatures
\pysigline{\sphinxbfcode{\sphinxupquote{field\_get\_buildings}}\sphinxbfcode{\sphinxupquote{\DUrole{p}{:}\DUrole{w}{ }str}}}
\pysigstopsignatures
\end{fulllineitems}

\index{field\_get\_sinks (pysewer.config.settings.Preporocessing attribute)@\spxentry{field\_get\_sinks}\spxextra{pysewer.config.settings.Preporocessing attribute}}

\begin{fulllineitems}
\phantomsection\label{\detokenize{pysewer:pysewer.config.settings.Preporocessing.field_get_sinks}}
\pysigstartsignatures
\pysigline{\sphinxbfcode{\sphinxupquote{field\_get\_sinks}}\sphinxbfcode{\sphinxupquote{\DUrole{p}{:}\DUrole{w}{ }str}}}
\pysigstopsignatures
\end{fulllineitems}

\index{max\_connection\_length (pysewer.config.settings.Preporocessing attribute)@\spxentry{max\_connection\_length}\spxextra{pysewer.config.settings.Preporocessing attribute}}

\begin{fulllineitems}
\phantomsection\label{\detokenize{pysewer:pysewer.config.settings.Preporocessing.max_connection_length}}
\pysigstartsignatures
\pysigline{\sphinxbfcode{\sphinxupquote{max\_connection\_length}}\sphinxbfcode{\sphinxupquote{\DUrole{p}{:}\DUrole{w}{ }int}}}
\pysigstopsignatures
\end{fulllineitems}

\index{pump\_penalty (pysewer.config.settings.Preporocessing attribute)@\spxentry{pump\_penalty}\spxextra{pysewer.config.settings.Preporocessing attribute}}

\begin{fulllineitems}
\phantomsection\label{\detokenize{pysewer:pysewer.config.settings.Preporocessing.pump_penalty}}
\pysigstartsignatures
\pysigline{\sphinxbfcode{\sphinxupquote{pump\_penalty}}\sphinxbfcode{\sphinxupquote{\DUrole{p}{:}\DUrole{w}{ }int}}}
\pysigstopsignatures
\end{fulllineitems}

\index{roads\_input\_data (pysewer.config.settings.Preporocessing attribute)@\spxentry{roads\_input\_data}\spxextra{pysewer.config.settings.Preporocessing attribute}}

\begin{fulllineitems}
\phantomsection\label{\detokenize{pysewer:pysewer.config.settings.Preporocessing.roads_input_data}}
\pysigstartsignatures
\pysigline{\sphinxbfcode{\sphinxupquote{roads\_input\_data}}\sphinxbfcode{\sphinxupquote{\DUrole{p}{:}\DUrole{w}{ }str\DUrole{w}{ }\DUrole{p}{|}\DUrole{w}{ }GeoDataFrame\DUrole{w}{ }\DUrole{p}{|}\DUrole{w}{ }None}}}
\pysigstopsignatures
\end{fulllineitems}

\index{value\_get\_buildings (pysewer.config.settings.Preporocessing attribute)@\spxentry{value\_get\_buildings}\spxextra{pysewer.config.settings.Preporocessing attribute}}

\begin{fulllineitems}
\phantomsection\label{\detokenize{pysewer:pysewer.config.settings.Preporocessing.value_get_buildings}}
\pysigstartsignatures
\pysigline{\sphinxbfcode{\sphinxupquote{value\_get\_buildings}}\sphinxbfcode{\sphinxupquote{\DUrole{p}{:}\DUrole{w}{ }str}}}
\pysigstopsignatures
\end{fulllineitems}

\index{value\_get\_sinks (pysewer.config.settings.Preporocessing attribute)@\spxentry{value\_get\_sinks}\spxextra{pysewer.config.settings.Preporocessing attribute}}

\begin{fulllineitems}
\phantomsection\label{\detokenize{pysewer:pysewer.config.settings.Preporocessing.value_get_sinks}}
\pysigstartsignatures
\pysigline{\sphinxbfcode{\sphinxupquote{value\_get\_sinks}}\sphinxbfcode{\sphinxupquote{\DUrole{p}{:}\DUrole{w}{ }str}}}
\pysigstopsignatures
\end{fulllineitems}


\end{fulllineitems}

\index{config\_to\_dataframe() (in module pysewer.config.settings)@\spxentry{config\_to\_dataframe()}\spxextra{in module pysewer.config.settings}}

\begin{fulllineitems}
\phantomsection\label{\detokenize{pysewer:pysewer.config.settings.config_to_dataframe}}
\pysigstartsignatures
\pysiglinewithargsret{\sphinxcode{\sphinxupquote{pysewer.config.settings.}}\sphinxbfcode{\sphinxupquote{config\_to\_dataframe}}}{\sphinxparam{\DUrole{n}{config}\DUrole{p}{:}\DUrole{w}{ }\DUrole{n}{Config}}}{{ $\rightarrow$ DataFrame}}
\pysigstopsignatures
\end{fulllineitems}

\index{deep\_merge() (in module pysewer.config.settings)@\spxentry{deep\_merge()}\spxextra{in module pysewer.config.settings}}

\begin{fulllineitems}
\phantomsection\label{\detokenize{pysewer:pysewer.config.settings.deep_merge}}
\pysigstartsignatures
\pysiglinewithargsret{\sphinxcode{\sphinxupquote{pysewer.config.settings.}}\sphinxbfcode{\sphinxupquote{deep\_merge}}}{\sphinxparam{\DUrole{n}{source}}\sphinxparamcomma \sphinxparam{\DUrole{n}{destination}}}{}
\pysigstopsignatures
\end{fulllineitems}

\index{flatten\_config() (in module pysewer.config.settings)@\spxentry{flatten\_config()}\spxextra{in module pysewer.config.settings}}

\begin{fulllineitems}
\phantomsection\label{\detokenize{pysewer:pysewer.config.settings.flatten_config}}
\pysigstartsignatures
\pysiglinewithargsret{\sphinxcode{\sphinxupquote{pysewer.config.settings.}}\sphinxbfcode{\sphinxupquote{flatten\_config}}}{\sphinxparam{\DUrole{n}{config\_dict}}\sphinxparamcomma \sphinxparam{\DUrole{n}{parent\_key}\DUrole{o}{=}\DUrole{default_value}{\textquotesingle{}\textquotesingle{}}}\sphinxparamcomma \sphinxparam{\DUrole{n}{sep}\DUrole{o}{=}\DUrole{default_value}{\textquotesingle{}\_\textquotesingle{}}}}{}
\pysigstopsignatures
\end{fulllineitems}

\index{load\_config() (in module pysewer.config.settings)@\spxentry{load\_config()}\spxextra{in module pysewer.config.settings}}

\begin{fulllineitems}
\phantomsection\label{\detokenize{pysewer:pysewer.config.settings.load_config}}
\pysigstartsignatures
\pysiglinewithargsret{\sphinxcode{\sphinxupquote{pysewer.config.settings.}}\sphinxbfcode{\sphinxupquote{load\_config}}}{\sphinxparam{\DUrole{n}{custom\_path}\DUrole{p}{:}\DUrole{w}{ }\DUrole{n}{str\DUrole{w}{ }\DUrole{p}{|}\DUrole{w}{ }None}\DUrole{w}{ }\DUrole{o}{=}\DUrole{w}{ }\DUrole{default_value}{None}}\sphinxparamcomma \sphinxparam{\DUrole{n}{custom\_setting\_dict}\DUrole{p}{:}\DUrole{w}{ }\DUrole{n}{dict\DUrole{w}{ }\DUrole{p}{|}\DUrole{w}{ }None}\DUrole{w}{ }\DUrole{o}{=}\DUrole{w}{ }\DUrole{default_value}{None}}}{{ $\rightarrow$ Config}}
\pysigstopsignatures
\sphinxAtStartPar
Load the default settings and override them with custom settings if needed.

\end{fulllineitems}

\index{load\_settings() (in module pysewer.config.settings)@\spxentry{load\_settings()}\spxextra{in module pysewer.config.settings}}

\begin{fulllineitems}
\phantomsection\label{\detokenize{pysewer:pysewer.config.settings.load_settings}}
\pysigstartsignatures
\pysiglinewithargsret{\sphinxcode{\sphinxupquote{pysewer.config.settings.}}\sphinxbfcode{\sphinxupquote{load\_settings}}}{\sphinxparam{\DUrole{n}{file\_path}\DUrole{p}{:}\DUrole{w}{ }\DUrole{n}{str\DUrole{w}{ }\DUrole{p}{|}\DUrole{w}{ }None}\DUrole{w}{ }\DUrole{o}{=}\DUrole{w}{ }\DUrole{default_value}{None}}}{}
\pysigstopsignatures
\end{fulllineitems}

\index{override\_setting\_to\_config() (in module pysewer.config.settings)@\spxentry{override\_setting\_to\_config()}\spxextra{in module pysewer.config.settings}}

\begin{fulllineitems}
\phantomsection\label{\detokenize{pysewer:pysewer.config.settings.override_setting_to_config}}
\pysigstartsignatures
\pysiglinewithargsret{\sphinxcode{\sphinxupquote{pysewer.config.settings.}}\sphinxbfcode{\sphinxupquote{override\_setting\_to\_config}}}{\sphinxparam{\DUrole{n}{custom\_path}\DUrole{p}{:}\DUrole{w}{ }\DUrole{n}{str\DUrole{w}{ }\DUrole{p}{|}\DUrole{w}{ }None}\DUrole{w}{ }\DUrole{o}{=}\DUrole{w}{ }\DUrole{default_value}{None}}\sphinxparamcomma \sphinxparam{\DUrole{n}{custom\_setting\_dict}\DUrole{p}{:}\DUrole{w}{ }\DUrole{n}{dict\DUrole{w}{ }\DUrole{p}{|}\DUrole{w}{ }None}\DUrole{w}{ }\DUrole{o}{=}\DUrole{w}{ }\DUrole{default_value}{None}}}{}
\pysigstopsignatures
\end{fulllineitems}

\index{override\_settings() (in module pysewer.config.settings)@\spxentry{override\_settings()}\spxextra{in module pysewer.config.settings}}

\begin{fulllineitems}
\phantomsection\label{\detokenize{pysewer:pysewer.config.settings.override_settings}}
\pysigstartsignatures
\pysiglinewithargsret{\sphinxcode{\sphinxupquote{pysewer.config.settings.}}\sphinxbfcode{\sphinxupquote{override\_settings}}}{\sphinxparam{\DUrole{n}{custom\_path}\DUrole{p}{:}\DUrole{w}{ }\DUrole{n}{str\DUrole{w}{ }\DUrole{p}{|}\DUrole{w}{ }None}\DUrole{w}{ }\DUrole{o}{=}\DUrole{w}{ }\DUrole{default_value}{None}}\sphinxparamcomma \sphinxparam{\DUrole{n}{custom\_setting\_dict}\DUrole{p}{:}\DUrole{w}{ }\DUrole{n}{dict\DUrole{w}{ }\DUrole{p}{|}\DUrole{w}{ }None}\DUrole{w}{ }\DUrole{o}{=}\DUrole{w}{ }\DUrole{default_value}{None}}}{}
\pysigstopsignatures
\sphinxAtStartPar
Override the settings with a custom settings file or a custom dictionary.

\end{fulllineitems}

\index{view\_default\_settings() (in module pysewer.config.settings)@\spxentry{view\_default\_settings()}\spxextra{in module pysewer.config.settings}}

\begin{fulllineitems}
\phantomsection\label{\detokenize{pysewer:pysewer.config.settings.view_default_settings}}
\pysigstartsignatures
\pysiglinewithargsret{\sphinxcode{\sphinxupquote{pysewer.config.settings.}}\sphinxbfcode{\sphinxupquote{view\_default\_settings}}}{}{}
\pysigstopsignatures
\end{fulllineitems}



\section{Export}
\label{\detokenize{pysewer:module-pysewer.export}}\label{\detokenize{pysewer:export}}\index{module@\spxentry{module}!pysewer.export@\spxentry{pysewer.export}}\index{pysewer.export@\spxentry{pysewer.export}!module@\spxentry{module}}
\sphinxAtStartPar
Since the profile and trench\_depth\_profile are list of tuples, they cannot be exported
to a shapefile or GeoPackage directly using the native GeoPandas function (to\_file).
This module contains functions to convert the list of tuples to JSON strings,
which can then be exported to a shapefile or GeoPackage.
The option saving the data as a parquet file is added.
\index{export\_sewer\_network() (in module pysewer.export)@\spxentry{export\_sewer\_network()}\spxextra{in module pysewer.export}}

\begin{fulllineitems}
\phantomsection\label{\detokenize{pysewer:pysewer.export.export_sewer_network}}
\pysigstartsignatures
\pysiglinewithargsret{\sphinxcode{\sphinxupquote{pysewer.export.}}\sphinxbfcode{\sphinxupquote{export\_sewer\_network}}}{\sphinxparam{\DUrole{n}{gdf}\DUrole{p}{:}\DUrole{w}{ }\DUrole{n}{GeoDataFrame}}\sphinxparamcomma \sphinxparam{\DUrole{n}{filepath}\DUrole{p}{:}\DUrole{w}{ }\DUrole{n}{str}}\sphinxparamcomma \sphinxparam{\DUrole{n}{file\_format}\DUrole{p}{:}\DUrole{w}{ }\DUrole{n}{str}\DUrole{w}{ }\DUrole{o}{=}\DUrole{w}{ }\DUrole{default_value}{\textquotesingle{}gpkg\textquotesingle{}}}}{}
\pysigstopsignatures
\sphinxAtStartPar
Export a sewer network GeoDataFrame to a file.
\begin{quote}\begin{description}
\sphinxlineitem{Parameters}\begin{itemize}
\item {} 
\sphinxAtStartPar
\sphinxstyleliteralstrong{\sphinxupquote{gdf}} (\sphinxstyleliteralemphasis{\sphinxupquote{gpd.GeoDataFrame}}) \textendash{} A GeoDataFrame containing the sewer network.

\item {} 
\sphinxAtStartPar
\sphinxstyleliteralstrong{\sphinxupquote{filepath}} (\sphinxstyleliteralemphasis{\sphinxupquote{str}}) \textendash{} The path to the file to which the sewer network should be exported.

\item {} 
\sphinxAtStartPar
\sphinxstyleliteralstrong{\sphinxupquote{file\_format}} (\sphinxstyleliteralemphasis{\sphinxupquote{str}}) \textendash{} The file format to which the sewer network should be exported. Default is ‘gpkg’ (GeoPackage).
Currently supported formats are ‘gpkg’ (GeoPackage), ‘shp’ (ESRI Shapefile) and Geoparquet ‘parquet’.

\end{itemize}

\sphinxlineitem{Raises}
\sphinxAtStartPar
\sphinxstyleliteralstrong{\sphinxupquote{ValueError}} \textendash{} If the file format is not supported.

\sphinxlineitem{Return type}
\sphinxAtStartPar
None

\end{description}\end{quote}

\end{fulllineitems}

\index{generate\_schema() (in module pysewer.export)@\spxentry{generate\_schema()}\spxextra{in module pysewer.export}}

\begin{fulllineitems}
\phantomsection\label{\detokenize{pysewer:pysewer.export.generate_schema}}
\pysigstartsignatures
\pysiglinewithargsret{\sphinxcode{\sphinxupquote{pysewer.export.}}\sphinxbfcode{\sphinxupquote{generate\_schema}}}{\sphinxparam{\DUrole{n}{gdf}\DUrole{p}{:}\DUrole{w}{ }\DUrole{n}{GeoDataFrame}}}{}
\pysigstopsignatures
\sphinxAtStartPar
Generate a schema based on the GeoDataFrame.
\begin{quote}\begin{description}
\sphinxlineitem{Parameters}
\sphinxAtStartPar
\sphinxstyleliteralstrong{\sphinxupquote{gdf}} (\sphinxstyleliteralemphasis{\sphinxupquote{gpd.GeoDataFrame}}) \textendash{} The GeoDataFrame to generate the schema from.

\sphinxlineitem{Returns}
\sphinxAtStartPar
The schema dictionary.

\sphinxlineitem{Return type}
\sphinxAtStartPar
dict

\end{description}\end{quote}

\end{fulllineitems}

\index{is\_list\_of\_tuples() (in module pysewer.export)@\spxentry{is\_list\_of\_tuples()}\spxextra{in module pysewer.export}}

\begin{fulllineitems}
\phantomsection\label{\detokenize{pysewer:pysewer.export.is_list_of_tuples}}
\pysigstartsignatures
\pysiglinewithargsret{\sphinxcode{\sphinxupquote{pysewer.export.}}\sphinxbfcode{\sphinxupquote{is\_list\_of\_tuples}}}{\sphinxparam{\DUrole{n}{column}}}{}
\pysigstopsignatures
\sphinxAtStartPar
Check if a pandas Series contains lists of tuples.

\end{fulllineitems}

\index{map\_dtype\_to\_fiona() (in module pysewer.export)@\spxentry{map\_dtype\_to\_fiona()}\spxextra{in module pysewer.export}}

\begin{fulllineitems}
\phantomsection\label{\detokenize{pysewer:pysewer.export.map_dtype_to_fiona}}
\pysigstartsignatures
\pysiglinewithargsret{\sphinxcode{\sphinxupquote{pysewer.export.}}\sphinxbfcode{\sphinxupquote{map\_dtype\_to\_fiona}}}{\sphinxparam{\DUrole{n}{dtype}}}{}
\pysigstopsignatures
\sphinxAtStartPar
Map pandas data type to Fiona/OGR data type.

\end{fulllineitems}

\index{tuple\_list\_to\_json() (in module pysewer.export)@\spxentry{tuple\_list\_to\_json()}\spxextra{in module pysewer.export}}

\begin{fulllineitems}
\phantomsection\label{\detokenize{pysewer:pysewer.export.tuple_list_to_json}}
\pysigstartsignatures
\pysiglinewithargsret{\sphinxcode{\sphinxupquote{pysewer.export.}}\sphinxbfcode{\sphinxupquote{tuple\_list\_to\_json}}}{\sphinxparam{\DUrole{n}{tuple\_list}}}{}
\pysigstopsignatures
\sphinxAtStartPar
Serialize a list of tuples to a JSON string.

\end{fulllineitems}

\index{write\_gdf\_to\_gpkg() (in module pysewer.export)@\spxentry{write\_gdf\_to\_gpkg()}\spxextra{in module pysewer.export}}

\begin{fulllineitems}
\phantomsection\label{\detokenize{pysewer:pysewer.export.write_gdf_to_gpkg}}
\pysigstartsignatures
\pysiglinewithargsret{\sphinxcode{\sphinxupquote{pysewer.export.}}\sphinxbfcode{\sphinxupquote{write\_gdf\_to\_gpkg}}}{\sphinxparam{\DUrole{n}{gdf}\DUrole{p}{:}\DUrole{w}{ }\DUrole{n}{GeoDataFrame}}\sphinxparamcomma \sphinxparam{\DUrole{n}{filepath}\DUrole{p}{:}\DUrole{w}{ }\DUrole{n}{str}}}{}
\pysigstopsignatures
\sphinxAtStartPar
Write a GeoDataFrame to a GeoPackage (GPKG) file using Fiona, converting lists of tuples to JSON strings.
\begin{quote}\begin{description}
\sphinxlineitem{Parameters}\begin{itemize}
\item {} 
\sphinxAtStartPar
\sphinxstyleliteralstrong{\sphinxupquote{gdf}} (\sphinxstyleliteralemphasis{\sphinxupquote{gpd.GeoDataFrame}}) \textendash{} The GeoDataFrame to be written to the GPKG file.

\item {} 
\sphinxAtStartPar
\sphinxstyleliteralstrong{\sphinxupquote{filepath}} (\sphinxstyleliteralemphasis{\sphinxupquote{str}}) \textendash{} The file path to the GPKG file.

\end{itemize}

\sphinxlineitem{Return type}
\sphinxAtStartPar
None

\end{description}\end{quote}
\subsubsection*{Notes}

\sphinxAtStartPar
This function converts any columns with list of tuples to JSON strings before writing to the GPKG file.

\end{fulllineitems}

\index{write\_gdf\_to\_shp() (in module pysewer.export)@\spxentry{write\_gdf\_to\_shp()}\spxextra{in module pysewer.export}}

\begin{fulllineitems}
\phantomsection\label{\detokenize{pysewer:pysewer.export.write_gdf_to_shp}}
\pysigstartsignatures
\pysiglinewithargsret{\sphinxcode{\sphinxupquote{pysewer.export.}}\sphinxbfcode{\sphinxupquote{write\_gdf\_to\_shp}}}{\sphinxparam{\DUrole{n}{gdf}\DUrole{p}{:}\DUrole{w}{ }\DUrole{n}{GeoDataFrame}}\sphinxparamcomma \sphinxparam{\DUrole{n}{filepath}\DUrole{p}{:}\DUrole{w}{ }\DUrole{n}{str}}}{}
\pysigstopsignatures
\sphinxAtStartPar
Write a GeoDataFrame to an ESRI Shapefile (SHP) file using Fiona, converting lists of tuples to JSON strings.
\begin{quote}\begin{description}
\sphinxlineitem{Parameters}\begin{itemize}
\item {} 
\sphinxAtStartPar
\sphinxstyleliteralstrong{\sphinxupquote{gdf}} (\sphinxstyleliteralemphasis{\sphinxupquote{gpd.GeoDataFrame}}) \textendash{} The GeoDataFrame to be written to the SHP file.

\item {} 
\sphinxAtStartPar
\sphinxstyleliteralstrong{\sphinxupquote{filepath}} (\sphinxstyleliteralemphasis{\sphinxupquote{str}}) \textendash{} The file path to save the SHP file.

\end{itemize}

\sphinxlineitem{Return type}
\sphinxAtStartPar
None

\end{description}\end{quote}
\subsubsection*{Notes}

\sphinxAtStartPar
This function converts any columns in the GeoDataFrame that contain lists of tuples to JSON strings before writing to the SHP file.

\end{fulllineitems}



\section{Helper}
\label{\detokenize{pysewer:module-pysewer.helper}}\label{\detokenize{pysewer:helper}}\index{module@\spxentry{module}!pysewer.helper@\spxentry{pysewer.helper}}\index{pysewer.helper@\spxentry{pysewer.helper}!module@\spxentry{module}}\index{ckdnearest() (in module pysewer.helper)@\spxentry{ckdnearest()}\spxextra{in module pysewer.helper}}

\begin{fulllineitems}
\phantomsection\label{\detokenize{pysewer:pysewer.helper.ckdnearest}}
\pysigstartsignatures
\pysiglinewithargsret{\sphinxcode{\sphinxupquote{pysewer.helper.}}\sphinxbfcode{\sphinxupquote{ckdnearest}}}{\sphinxparam{\DUrole{n}{gdfA}\DUrole{p}{:}\DUrole{w}{ }\DUrole{n}{GeoDataFrame}}\sphinxparamcomma \sphinxparam{\DUrole{n}{gdfB}\DUrole{p}{:}\DUrole{w}{ }\DUrole{n}{GeoDataFrame}}\sphinxparamcomma \sphinxparam{\DUrole{n}{gdfB\_cols}\DUrole{o}{=}\DUrole{default_value}{{[}\textquotesingle{}closest\_edge\textquotesingle{}{]}}}}{{ $\rightarrow$ GeoDataFrame}}
\pysigstopsignatures
\sphinxAtStartPar
Returns a GeoDataFrame containing the closest geometry and attributes from gdfB to each geometry in gdfA.

\end{fulllineitems}

\index{get\_closest\_edge() (in module pysewer.helper)@\spxentry{get\_closest\_edge()}\spxextra{in module pysewer.helper}}

\begin{fulllineitems}
\phantomsection\label{\detokenize{pysewer:pysewer.helper.get_closest_edge}}
\pysigstartsignatures
\pysiglinewithargsret{\sphinxcode{\sphinxupquote{pysewer.helper.}}\sphinxbfcode{\sphinxupquote{get\_closest\_edge}}}{\sphinxparam{\DUrole{n}{G}\DUrole{p}{:}\DUrole{w}{ }\DUrole{n}{Graph}}\sphinxparamcomma \sphinxparam{\DUrole{n}{point}\DUrole{p}{:}\DUrole{w}{ }\DUrole{n}{Point}}}{{ $\rightarrow$ tuple}}
\pysigstopsignatures
\sphinxAtStartPar
Returns the closest edge to a given point in a networkx graph.


\subsection{Parameters:}
\label{\detokenize{pysewer:parameters}}\begin{description}
\sphinxlineitem{G}{[}networkx.Graph{]}
\sphinxAtStartPar
The graph to search for the closest edge.

\sphinxlineitem{point}{[}shapely.geometry.Point{]}
\sphinxAtStartPar
The point to search for the closest edge.

\end{description}


\subsection{Returns:}
\label{\detokenize{pysewer:returns}}\begin{description}
\sphinxlineitem{closest\_edge}{[}tuple{]}
\sphinxAtStartPar
A tuple representing the closest edge to the given point.

\end{description}

\end{fulllineitems}

\index{get\_closest\_edge\_multiple() (in module pysewer.helper)@\spxentry{get\_closest\_edge\_multiple()}\spxextra{in module pysewer.helper}}

\begin{fulllineitems}
\phantomsection\label{\detokenize{pysewer:pysewer.helper.get_closest_edge_multiple}}
\pysigstartsignatures
\pysiglinewithargsret{\sphinxcode{\sphinxupquote{pysewer.helper.}}\sphinxbfcode{\sphinxupquote{get\_closest\_edge\_multiple}}}{\sphinxparam{\DUrole{n}{G}\DUrole{p}{:}\DUrole{w}{ }\DUrole{n}{Graph}}\sphinxparamcomma \sphinxparam{\DUrole{n}{list\_of\_points}\DUrole{p}{:}\DUrole{w}{ }\DUrole{n}{list}}}{{ $\rightarrow$ list}}
\pysigstopsignatures
\sphinxAtStartPar
Returns a list of the closest edges in a networkx graph to a list of points.
\begin{quote}\begin{description}
\sphinxlineitem{Parameters}\begin{itemize}
\item {} 
\sphinxAtStartPar
\sphinxstyleliteralstrong{\sphinxupquote{G}} (\sphinxstyleliteralemphasis{\sphinxupquote{networkx.Graph}}) \textendash{} A networkx graph object.

\item {} 
\sphinxAtStartPar
\sphinxstyleliteralstrong{\sphinxupquote{list\_of\_points}} (\sphinxstyleliteralemphasis{\sphinxupquote{list}}) \textendash{} A list of shapely Point objects.

\end{itemize}

\sphinxlineitem{Returns}
\sphinxAtStartPar
A list of shapely LineString objects representing the closest edges to the input points.

\sphinxlineitem{Return type}
\sphinxAtStartPar
list

\end{description}\end{quote}

\end{fulllineitems}

\index{get\_edge\_gdf() (in module pysewer.helper)@\spxentry{get\_edge\_gdf()}\spxextra{in module pysewer.helper}}

\begin{fulllineitems}
\phantomsection\label{\detokenize{pysewer:pysewer.helper.get_edge_gdf}}
\pysigstartsignatures
\pysiglinewithargsret{\sphinxcode{\sphinxupquote{pysewer.helper.}}\sphinxbfcode{\sphinxupquote{get\_edge\_gdf}}}{\sphinxparam{\DUrole{n}{G}\DUrole{p}{:}\DUrole{w}{ }\DUrole{n}{Graph}}\sphinxparamcomma \sphinxparam{\DUrole{n}{field}\DUrole{p}{:}\DUrole{w}{ }\DUrole{n}{str\DUrole{w}{ }\DUrole{p}{|}\DUrole{w}{ }None}\DUrole{w}{ }\DUrole{o}{=}\DUrole{w}{ }\DUrole{default_value}{None}}\sphinxparamcomma \sphinxparam{\DUrole{n}{value}\DUrole{p}{:}\DUrole{w}{ }\DUrole{n}{any\DUrole{w}{ }\DUrole{p}{|}\DUrole{w}{ }None}\DUrole{w}{ }\DUrole{o}{=}\DUrole{w}{ }\DUrole{default_value}{None}}\sphinxparamcomma \sphinxparam{\DUrole{n}{detailed}\DUrole{p}{:}\DUrole{w}{ }\DUrole{n}{bool}\DUrole{w}{ }\DUrole{o}{=}\DUrole{w}{ }\DUrole{default_value}{False}}}{{ $\rightarrow$ GeoDataFrame}}
\pysigstopsignatures
\sphinxAtStartPar
Returns a GeoDataFrame of edges in a networkx graph that match a given field and value.
\begin{quote}\begin{description}
\sphinxlineitem{Parameters}\begin{itemize}
\item {} 
\sphinxAtStartPar
\sphinxstyleliteralstrong{\sphinxupquote{G}} (\sphinxstyleliteralemphasis{\sphinxupquote{networkx.Graph}}) \textendash{} The graph to extract edges from.

\item {} 
\sphinxAtStartPar
\sphinxstyleliteralstrong{\sphinxupquote{field}} (\sphinxstyleliteralemphasis{\sphinxupquote{str}}\sphinxstyleliteralemphasis{\sphinxupquote{, }}\sphinxstyleliteralemphasis{\sphinxupquote{optional}}) \textendash{} The edge attribute to filter on.

\item {} 
\sphinxAtStartPar
\sphinxstyleliteralstrong{\sphinxupquote{value}} (\sphinxstyleliteralemphasis{\sphinxupquote{any}}\sphinxstyleliteralemphasis{\sphinxupquote{, }}\sphinxstyleliteralemphasis{\sphinxupquote{optional}}) \textendash{} The value to filter the edge attribute on.

\item {} 
\sphinxAtStartPar
\sphinxstyleliteralstrong{\sphinxupquote{detailed}} (\sphinxstyleliteralemphasis{\sphinxupquote{bool}}\sphinxstyleliteralemphasis{\sphinxupquote{, }}\sphinxstyleliteralemphasis{\sphinxupquote{optional}}) \textendash{} If True, returns a detailed GeoDataFrame with all edge attributes. If False, returns a simplified GeoDataFrame with only the edge geometry.

\end{itemize}

\sphinxlineitem{Returns}
\sphinxAtStartPar
A GeoDataFrame of edges that match the given field and value.

\sphinxlineitem{Return type}
\sphinxAtStartPar
gpd.GeoDataFrame

\end{description}\end{quote}

\end{fulllineitems}

\index{get\_edge\_keys() (in module pysewer.helper)@\spxentry{get\_edge\_keys()}\spxextra{in module pysewer.helper}}

\begin{fulllineitems}
\phantomsection\label{\detokenize{pysewer:pysewer.helper.get_edge_keys}}
\pysigstartsignatures
\pysiglinewithargsret{\sphinxcode{\sphinxupquote{pysewer.helper.}}\sphinxbfcode{\sphinxupquote{get\_edge\_keys}}}{\sphinxparam{\DUrole{n}{G}}\sphinxparamcomma \sphinxparam{\DUrole{n}{field}\DUrole{o}{=}\DUrole{default_value}{None}}\sphinxparamcomma \sphinxparam{\DUrole{n}{value}\DUrole{o}{=}\DUrole{default_value}{None}}}{}
\pysigstopsignatures
\sphinxAtStartPar
Returns a list of edge keys (tuples of nodes) for the given graph \sphinxtitleref{G} that have an edge attribute \sphinxtitleref{field} with value \sphinxtitleref{value}.


\subsection{Parameters:}
\label{\detokenize{pysewer:id1}}\begin{description}
\sphinxlineitem{G}{[}networkx.Graph{]}
\sphinxAtStartPar
The graph to search for edges.

\sphinxlineitem{field}{[}str, optional{]}
\sphinxAtStartPar
The name of the edge attribute to filter on.

\sphinxlineitem{value}{[}any, optional{]}
\sphinxAtStartPar
The value of the edge attribute to filter on.

\end{description}


\subsection{Returns:}
\label{\detokenize{pysewer:id2}}\begin{description}
\sphinxlineitem{list of tuples}
\sphinxAtStartPar
A list of edge keys (tuples of nodes) that have an edge attribute \sphinxtitleref{field} with value \sphinxtitleref{value}.

\end{description}

\end{fulllineitems}

\index{get\_mean\_slope() (in module pysewer.helper)@\spxentry{get\_mean\_slope()}\spxextra{in module pysewer.helper}}

\begin{fulllineitems}
\phantomsection\label{\detokenize{pysewer:pysewer.helper.get_mean_slope}}
\pysigstartsignatures
\pysiglinewithargsret{\sphinxcode{\sphinxupquote{pysewer.helper.}}\sphinxbfcode{\sphinxupquote{get\_mean\_slope}}}{\sphinxparam{\DUrole{n}{G}\DUrole{p}{:}\DUrole{w}{ }\DUrole{n}{MultiDiGraph}}\sphinxparamcomma \sphinxparam{\DUrole{n}{upstream}\DUrole{p}{:}\DUrole{w}{ }\DUrole{n}{int}}\sphinxparamcomma \sphinxparam{\DUrole{n}{downstream}\DUrole{p}{:}\DUrole{w}{ }\DUrole{n}{int}}\sphinxparamcomma \sphinxparam{\DUrole{n}{us\_td}\DUrole{p}{:}\DUrole{w}{ }\DUrole{n}{float}}\sphinxparamcomma \sphinxparam{\DUrole{n}{ds\_td}\DUrole{p}{:}\DUrole{w}{ }\DUrole{n}{float}}}{{ $\rightarrow$ float}}
\pysigstopsignatures
\sphinxAtStartPar
Calculates the mean slope of a path between two nodes in a graph.


\subsection{Parameters:}
\label{\detokenize{pysewer:id3}}\begin{description}
\sphinxlineitem{G}{[}networkx.MultiDiGraph{]}
\sphinxAtStartPar
A directed graph object.

\sphinxlineitem{upstream}{[}int{]}
\sphinxAtStartPar
The upstream node ID.

\sphinxlineitem{downstream}{[}int{]}
\sphinxAtStartPar
The downstream node ID.

\sphinxlineitem{us\_td}{[}float{]}
\sphinxAtStartPar
The upstream node topographic elevation.

\sphinxlineitem{ds\_td}{[}float{]}
\sphinxAtStartPar
The downstream node topographic elevation.

\end{description}


\subsection{Returns:}
\label{\detokenize{pysewer:id4}}\begin{description}
\sphinxlineitem{float}
\sphinxAtStartPar
The mean slope of the path between the upstream and downstream nodes.

\end{description}

\end{fulllineitems}

\index{get\_node\_gdf() (in module pysewer.helper)@\spxentry{get\_node\_gdf()}\spxextra{in module pysewer.helper}}

\begin{fulllineitems}
\phantomsection\label{\detokenize{pysewer:pysewer.helper.get_node_gdf}}
\pysigstartsignatures
\pysiglinewithargsret{\sphinxcode{\sphinxupquote{pysewer.helper.}}\sphinxbfcode{\sphinxupquote{get\_node\_gdf}}}{\sphinxparam{\DUrole{n}{G}\DUrole{p}{:}\DUrole{w}{ }\DUrole{n}{Graph}}\sphinxparamcomma \sphinxparam{\DUrole{n}{field}\DUrole{o}{=}\DUrole{default_value}{None}}\sphinxparamcomma \sphinxparam{\DUrole{n}{value}\DUrole{o}{=}\DUrole{default_value}{None}}}{{ $\rightarrow$ GeoDataFrame}}
\pysigstopsignatures
\sphinxAtStartPar
Returns a GeoDataFrame of nodes in the graph that match the specified field and value.
\begin{quote}\begin{description}
\sphinxlineitem{Parameters}\begin{itemize}
\item {} 
\sphinxAtStartPar
\sphinxstyleliteralstrong{\sphinxupquote{G}} (\sphinxstyleliteralemphasis{\sphinxupquote{nx.Graph}}) \textendash{} The input graph.

\item {} 
\sphinxAtStartPar
\sphinxstyleliteralstrong{\sphinxupquote{field}} (\sphinxstyleliteralemphasis{\sphinxupquote{str}}\sphinxstyleliteralemphasis{\sphinxupquote{, }}\sphinxstyleliteralemphasis{\sphinxupquote{optional}}) \textendash{} The node attribute to filter on.

\item {} 
\sphinxAtStartPar
\sphinxstyleliteralstrong{\sphinxupquote{value}} (\sphinxstyleliteralemphasis{\sphinxupquote{str}}\sphinxstyleliteralemphasis{\sphinxupquote{, }}\sphinxstyleliteralemphasis{\sphinxupquote{optional}}) \textendash{} The value to filter on.

\end{itemize}

\sphinxlineitem{Returns}
\sphinxAtStartPar
\sphinxstylestrong{gdf} \textendash{} A GeoDataFrame of nodes that match the specified field and value.

\sphinxlineitem{Return type}
\sphinxAtStartPar
gpd.GeoDataFrame

\sphinxlineitem{Raises}
\sphinxAtStartPar
\sphinxstyleliteralstrong{\sphinxupquote{ValueError}} \textendash{} If no geometry column is found in the GeoDataFrame.

\end{description}\end{quote}
\subsubsection*{Notes}

\sphinxAtStartPar
This function filters the nodes in the input graph based on the specified field and value, and returns a GeoDataFrame
containing the filtered nodes and their attributes. The GeoDataFrame is created using the coordinates of the filtered
nodes and their attributes.

\end{fulllineitems}

\index{get\_node\_keys() (in module pysewer.helper)@\spxentry{get\_node\_keys()}\spxextra{in module pysewer.helper}}

\begin{fulllineitems}
\phantomsection\label{\detokenize{pysewer:pysewer.helper.get_node_keys}}
\pysigstartsignatures
\pysiglinewithargsret{\sphinxcode{\sphinxupquote{pysewer.helper.}}\sphinxbfcode{\sphinxupquote{get\_node\_keys}}}{\sphinxparam{\DUrole{n}{G}\DUrole{p}{:}\DUrole{w}{ }\DUrole{n}{Graph}}\sphinxparamcomma \sphinxparam{\DUrole{n}{field}\DUrole{p}{:}\DUrole{w}{ }\DUrole{n}{str\DUrole{w}{ }\DUrole{p}{|}\DUrole{w}{ }None}\DUrole{w}{ }\DUrole{o}{=}\DUrole{w}{ }\DUrole{default_value}{None}}\sphinxparamcomma \sphinxparam{\DUrole{n}{value}\DUrole{p}{:}\DUrole{w}{ }\DUrole{n}{str\DUrole{w}{ }\DUrole{p}{|}\DUrole{w}{ }None}\DUrole{w}{ }\DUrole{o}{=}\DUrole{w}{ }\DUrole{default_value}{None}}}{}
\pysigstopsignatures
\sphinxAtStartPar
Returns a list of keys for nodes in the graph \sphinxtitleref{G} that have a specified \sphinxtitleref{field} with a specified \sphinxtitleref{value}.


\subsection{Parameters:}
\label{\detokenize{pysewer:id5}}\begin{description}
\sphinxlineitem{G}{[}networkx.Graph{]}
\sphinxAtStartPar
The graph to search for nodes.

\sphinxlineitem{field}{[}str, optional{]}
\sphinxAtStartPar
The field to search for in the node data dictionary. If None, all nodes are returned.

\sphinxlineitem{value}{[}str{]}
\sphinxAtStartPar
The value to search for in the specified \sphinxtitleref{field}. If None, all nodes with the specified \sphinxtitleref{field} are returned.

\end{description}


\subsection{Returns:}
\label{\detokenize{pysewer:id6}}\begin{description}
\sphinxlineitem{list}
\sphinxAtStartPar
A list of keys for nodes in the graph \sphinxtitleref{G} that have a specified \sphinxtitleref{field} with a specified \sphinxtitleref{value}.

\end{description}

\end{fulllineitems}

\index{get\_path\_distance() (in module pysewer.helper)@\spxentry{get\_path\_distance()}\spxextra{in module pysewer.helper}}

\begin{fulllineitems}
\phantomsection\label{\detokenize{pysewer:pysewer.helper.get_path_distance}}
\pysigstartsignatures
\pysiglinewithargsret{\sphinxcode{\sphinxupquote{pysewer.helper.}}\sphinxbfcode{\sphinxupquote{get\_path\_distance}}}{\sphinxparam{\DUrole{n}{detailed\_path}\DUrole{p}{:}\DUrole{w}{ }\DUrole{n}{List\DUrole{p}{{[}}tuple\DUrole{p}{{]}}}}}{{ $\rightarrow$ float}}
\pysigstopsignatures
\sphinxAtStartPar
Calculates the total distance of a path given a list of detailed path coordinates.
\begin{quote}\begin{description}
\sphinxlineitem{Parameters}
\sphinxAtStartPar
\sphinxstyleliteralstrong{\sphinxupquote{detailed\_path}} (\sphinxstyleliteralemphasis{\sphinxupquote{List}}\sphinxstyleliteralemphasis{\sphinxupquote{{[}}}\sphinxstyleliteralemphasis{\sphinxupquote{tuple}}\sphinxstyleliteralemphasis{\sphinxupquote{{]}}}) \textendash{} A list of tuples representing the detailed path coordinates.

\sphinxlineitem{Returns}
\sphinxAtStartPar
The total distance of the path.

\sphinxlineitem{Return type}
\sphinxAtStartPar
float

\end{description}\end{quote}
\subsubsection*{Examples}

\begin{sphinxVerbatim}[commandchars=\\\{\}]
\PYG{g+gp}{\PYGZgt{}\PYGZgt{}\PYGZgt{} }\PYG{n}{get\PYGZus{}path\PYGZus{}distance}\PYG{p}{(}\PYG{p}{[}\PYG{p}{(}\PYG{l+m+mi}{0}\PYG{p}{,} \PYG{l+m+mi}{0}\PYG{p}{)}\PYG{p}{,} \PYG{p}{(}\PYG{l+m+mi}{3}\PYG{p}{,} \PYG{l+m+mi}{4}\PYG{p}{)}\PYG{p}{,} \PYG{p}{(}\PYG{l+m+mi}{7}\PYG{p}{,} \PYG{l+m+mi}{1}\PYG{p}{)}\PYG{p}{]}\PYG{p}{)}
\PYG{g+go}{9.848857801796104}
\end{sphinxVerbatim}

\end{fulllineitems}

\index{get\_path\_gdf() (in module pysewer.helper)@\spxentry{get\_path\_gdf()}\spxextra{in module pysewer.helper}}

\begin{fulllineitems}
\phantomsection\label{\detokenize{pysewer:pysewer.helper.get_path_gdf}}
\pysigstartsignatures
\pysiglinewithargsret{\sphinxcode{\sphinxupquote{pysewer.helper.}}\sphinxbfcode{\sphinxupquote{get\_path\_gdf}}}{\sphinxparam{\DUrole{n}{G}}\sphinxparamcomma \sphinxparam{\DUrole{n}{upstream}}\sphinxparamcomma \sphinxparam{\DUrole{n}{downstream}}}{}
\pysigstopsignatures
\sphinxAtStartPar
Returns a GeoDataFrame containing the geometry of the shortest path between two nodes in a graph.


\subsection{Parameters:}
\label{\detokenize{pysewer:id7}}\begin{description}
\sphinxlineitem{G}{[}networkx.Graph{]}
\sphinxAtStartPar
The graph to find the shortest path in.

\sphinxlineitem{upstream}{[}int{]}
\sphinxAtStartPar
The starting node of the path.

\sphinxlineitem{downstream}{[}int{]}
\sphinxAtStartPar
The ending node of the path.

\end{description}


\subsection{Returns:}
\label{\detokenize{pysewer:id8}}\begin{description}
\sphinxlineitem{gpd.GeoDataFrame}
\sphinxAtStartPar
A GeoDataFrame containing the geometry of the shortest path between the upstream and downstream nodes.

\end{description}

\end{fulllineitems}

\index{get\_sewer\_info() (in module pysewer.helper)@\spxentry{get\_sewer\_info()}\spxextra{in module pysewer.helper}}

\begin{fulllineitems}
\phantomsection\label{\detokenize{pysewer:pysewer.helper.get_sewer_info}}
\pysigstartsignatures
\pysiglinewithargsret{\sphinxcode{\sphinxupquote{pysewer.helper.}}\sphinxbfcode{\sphinxupquote{get\_sewer\_info}}}{\sphinxparam{\DUrole{n}{G}}}{}
\pysigstopsignatures
\sphinxAtStartPar
Returns a dictionary with information about the sewer network.
\begin{quote}\begin{description}
\sphinxlineitem{Parameters}
\sphinxAtStartPar
\sphinxstyleliteralstrong{\sphinxupquote{G}} (\sphinxstyleliteralemphasis{\sphinxupquote{networkx.Graph}}) \textendash{} The sewer network graph.

\sphinxlineitem{Returns}
\sphinxAtStartPar
A dictionary containing the following information:
\sphinxhyphen{} Total Buildings: Total number of buildings in the network.
\sphinxhyphen{} Pressurized Sewer Length {[}m{]}: Total length of pressurized sewers in meters.
\sphinxhyphen{} Gravity Sewer Length {[}m{]}: Total length of gravity sewers in meters.
\sphinxhyphen{} Lifting Stations: Total number of lifting stations in the network.
\sphinxhyphen{} Pumping Stations: Total number of pumping stations in the network (excluding those located in buildings).
\sphinxhyphen{} Private Pumps: Total number of pumps located in buildings.

\sphinxlineitem{Return type}
\sphinxAtStartPar
dict

\end{description}\end{quote}

\end{fulllineitems}

\index{get\_upstream\_nodes() (in module pysewer.helper)@\spxentry{get\_upstream\_nodes()}\spxextra{in module pysewer.helper}}

\begin{fulllineitems}
\phantomsection\label{\detokenize{pysewer:pysewer.helper.get_upstream_nodes}}
\pysigstartsignatures
\pysiglinewithargsret{\sphinxcode{\sphinxupquote{pysewer.helper.}}\sphinxbfcode{\sphinxupquote{get\_upstream\_nodes}}}{\sphinxparam{\DUrole{n}{G}\DUrole{p}{:}\DUrole{w}{ }\DUrole{n}{DiGraph}}\sphinxparamcomma \sphinxparam{\DUrole{n}{start\_node}}\sphinxparamcomma \sphinxparam{\DUrole{n}{field}\DUrole{p}{:}\DUrole{w}{ }\DUrole{n}{str}}\sphinxparamcomma \sphinxparam{\DUrole{n}{value}\DUrole{p}{:}\DUrole{w}{ }\DUrole{n}{str}}}{{ $\rightarrow$ List}}
\pysigstopsignatures
\sphinxAtStartPar
Returns a list of all upstream nodes in a directed graph \sphinxtitleref{G} that have a node attribute \sphinxtitleref{field} with value \sphinxtitleref{value},
starting from \sphinxtitleref{start\_node} and traversing the graph in reverse order using a breadth\sphinxhyphen{}first search algorithm.
\begin{quote}\begin{description}
\sphinxlineitem{Parameters}\begin{itemize}
\item {} 
\sphinxAtStartPar
\sphinxstyleliteralstrong{\sphinxupquote{G}} (\sphinxstyleliteralemphasis{\sphinxupquote{networkx.DiGraph}}) \textendash{} The directed graph to traverse.

\item {} 
\sphinxAtStartPar
\sphinxstyleliteralstrong{\sphinxupquote{start\_node}} (\sphinxstyleliteralemphasis{\sphinxupquote{hashable}}) \textendash{} The node to start the traversal from.

\item {} 
\sphinxAtStartPar
\sphinxstyleliteralstrong{\sphinxupquote{field}} (\sphinxstyleliteralemphasis{\sphinxupquote{str}}) \textendash{} The name of the node attribute to filter by.

\item {} 
\sphinxAtStartPar
\sphinxstyleliteralstrong{\sphinxupquote{value}} (\sphinxstyleliteralemphasis{\sphinxupquote{Any}}) \textendash{} The value of the node attribute to filter by.

\end{itemize}

\sphinxlineitem{Returns}
\sphinxAtStartPar
A list of all upstream nodes in \sphinxtitleref{G} that have a node attribute \sphinxtitleref{field} with value \sphinxtitleref{value}.

\sphinxlineitem{Return type}
\sphinxAtStartPar
List

\end{description}\end{quote}

\end{fulllineitems}

\index{remove\_third\_dimension() (in module pysewer.helper)@\spxentry{remove\_third\_dimension()}\spxextra{in module pysewer.helper}}

\begin{fulllineitems}
\phantomsection\label{\detokenize{pysewer:pysewer.helper.remove_third_dimension}}
\pysigstartsignatures
\pysiglinewithargsret{\sphinxcode{\sphinxupquote{pysewer.helper.}}\sphinxbfcode{\sphinxupquote{remove\_third\_dimension}}}{\sphinxparam{\DUrole{n}{geom}}}{}
\pysigstopsignatures
\sphinxAtStartPar
remove the third dimension of a shapely geometry

\end{fulllineitems}



\section{Optimization}
\label{\detokenize{pysewer:module-pysewer.optimization}}\label{\detokenize{pysewer:optimization}}\index{module@\spxentry{module}!pysewer.optimization@\spxentry{pysewer.optimization}}\index{pysewer.optimization@\spxentry{pysewer.optimization}!module@\spxentry{module}}\index{calculate\_hydraulic\_parameters() (in module pysewer.optimization)@\spxentry{calculate\_hydraulic\_parameters()}\spxextra{in module pysewer.optimization}}

\begin{fulllineitems}
\phantomsection\label{\detokenize{pysewer:pysewer.optimization.calculate_hydraulic_parameters}}
\pysigstartsignatures
\pysiglinewithargsret{\sphinxcode{\sphinxupquote{pysewer.optimization.}}\sphinxbfcode{\sphinxupquote{calculate\_hydraulic\_parameters}}}{\sphinxparam{\DUrole{n}{G}}\sphinxparamcomma \sphinxparam{\DUrole{n}{sinks}\DUrole{p}{:}\DUrole{w}{ }\DUrole{n}{list}}\sphinxparamcomma \sphinxparam{\DUrole{n}{pressurized\_diameter}\DUrole{p}{:}\DUrole{w}{ }\DUrole{n}{float}\DUrole{w}{ }\DUrole{o}{=}\DUrole{w}{ }\DUrole{default_value}{0.2}}\sphinxparamcomma \sphinxparam{\DUrole{n}{diameters}\DUrole{p}{:}\DUrole{w}{ }\DUrole{n}{List\DUrole{p}{{[}}float\DUrole{p}{{]}}}\DUrole{w}{ }\DUrole{o}{=}\DUrole{w}{ }\DUrole{default_value}{{[}0.1, 0.15, 0.2, 0.25, 0.3, 0.4{]}}}\sphinxparamcomma \sphinxparam{\DUrole{n}{roughness}\DUrole{p}{:}\DUrole{w}{ }\DUrole{n}{float}\DUrole{w}{ }\DUrole{o}{=}\DUrole{w}{ }\DUrole{default_value}{0.013}}\sphinxparamcomma \sphinxparam{\DUrole{n}{include\_private\_sewer}\DUrole{p}{:}\DUrole{w}{ }\DUrole{n}{bool}\DUrole{w}{ }\DUrole{o}{=}\DUrole{w}{ }\DUrole{default_value}{True}}}{}
\pysigstopsignatures
\sphinxAtStartPar
Calculates hydraulic parameters for a sewer network graph.
\begin{quote}\begin{description}
\sphinxlineitem{Parameters}\begin{itemize}
\item {} 
\sphinxAtStartPar
\sphinxstyleliteralstrong{\sphinxupquote{G}} (\sphinxstyleliteralemphasis{\sphinxupquote{networkx.Graph}}) \textendash{} The sewer network graph.

\item {} 
\sphinxAtStartPar
\sphinxstyleliteralstrong{\sphinxupquote{sinks}} (\sphinxstyleliteralemphasis{\sphinxupquote{list}}) \textendash{} A list of sink nodes in the graph.

\item {} 
\sphinxAtStartPar
\sphinxstyleliteralstrong{\sphinxupquote{pressurized\_diameter}} (\sphinxstyleliteralemphasis{\sphinxupquote{float}}) \textendash{} The diameter of pressurized pipes in the network.

\item {} 
\sphinxAtStartPar
\sphinxstyleliteralstrong{\sphinxupquote{diameters}} (\sphinxstyleliteralemphasis{\sphinxupquote{list}}) \textendash{} A list of available pipe diameters.

\item {} 
\sphinxAtStartPar
\sphinxstyleliteralstrong{\sphinxupquote{roughness}} (\sphinxstyleliteralemphasis{\sphinxupquote{float}}) \textendash{} The roughness coefficient of the pipes.

\item {} 
\sphinxAtStartPar
\sphinxstyleliteralstrong{\sphinxupquote{include\_private\_sewer}} (\sphinxstyleliteralemphasis{\sphinxupquote{bool}}\sphinxstyleliteralemphasis{\sphinxupquote{, }}\sphinxstyleliteralemphasis{\sphinxupquote{optional}}) \textendash{} Whether to include private sewer connections in the graph, by default True.

\end{itemize}

\sphinxlineitem{Returns}
\sphinxAtStartPar
The sewer network graph with updated hydraulic parameters.

\sphinxlineitem{Return type}
\sphinxAtStartPar
networkx.Graph

\end{description}\end{quote}
\subsubsection*{Notes}

\sphinxAtStartPar
This function places pumps/lifting stations on linear sections between road junctions.
Three cases are possible:
1. Terrain does not allow for gravity flow to the downstream node (this check uses the “needs\_pump” attribute from the preprocessing
to reduce computational load) \sphinxhyphen{}\textgreater{} place pump
2. Terrain does not require pump but lowest inflow trench depth is too low for gravitational flow \sphinxhyphen{}\textgreater{} place lifting station
3. Gravity flow is possible within given constraints

\end{fulllineitems}

\index{estimate\_peakflow() (in module pysewer.optimization)@\spxentry{estimate\_peakflow()}\spxextra{in module pysewer.optimization}}

\begin{fulllineitems}
\phantomsection\label{\detokenize{pysewer:pysewer.optimization.estimate_peakflow}}
\pysigstartsignatures
\pysiglinewithargsret{\sphinxcode{\sphinxupquote{pysewer.optimization.}}\sphinxbfcode{\sphinxupquote{estimate\_peakflow}}}{\sphinxparam{\DUrole{n}{G}\DUrole{p}{:}\DUrole{w}{ }\DUrole{n}{Graph}}\sphinxparamcomma \sphinxparam{\DUrole{n}{inhabitants\_dwelling}\DUrole{p}{:}\DUrole{w}{ }\DUrole{n}{int}\DUrole{w}{ }\DUrole{o}{=}\DUrole{w}{ }\DUrole{default_value}{3}}\sphinxparamcomma \sphinxparam{\DUrole{n}{daily\_wastewater\_person}\DUrole{p}{:}\DUrole{w}{ }\DUrole{n}{float}\DUrole{w}{ }\DUrole{o}{=}\DUrole{w}{ }\DUrole{default_value}{0.2}}\sphinxparamcomma \sphinxparam{\DUrole{n}{peak\_factor}\DUrole{p}{:}\DUrole{w}{ }\DUrole{n}{float}\DUrole{w}{ }\DUrole{o}{=}\DUrole{w}{ }\DUrole{default_value}{2.3}}}{}
\pysigstopsignatures
\sphinxAtStartPar
Estimate the peakflow in m\(\sp{\text{3}}\)/s for a node n in Graph G.
\begin{quote}\begin{description}
\sphinxlineitem{Parameters}\begin{itemize}
\item {} 
\sphinxAtStartPar
\sphinxstyleliteralstrong{\sphinxupquote{G}} (\sphinxstyleliteralemphasis{\sphinxupquote{networkx.Graph}}) \textendash{} The graph to estimate peakflow for.

\item {} 
\sphinxAtStartPar
\sphinxstyleliteralstrong{\sphinxupquote{inhabitants\_dwelling}} (\sphinxstyleliteralemphasis{\sphinxupquote{int}}) \textendash{} The number of inhabitants per dwelling.

\item {} 
\sphinxAtStartPar
\sphinxstyleliteralstrong{\sphinxupquote{daily\_wastewater\_person}} (\sphinxstyleliteralemphasis{\sphinxupquote{float}}) \textendash{} The daily wastewater generated per person in m\(\sp{\text{3}}\).

\item {} 
\sphinxAtStartPar
\sphinxstyleliteralstrong{\sphinxupquote{peak\_factor}} (\sphinxstyleliteralemphasis{\sphinxupquote{float}}\sphinxstyleliteralemphasis{\sphinxupquote{, }}\sphinxstyleliteralemphasis{\sphinxupquote{optional}}) \textendash{} The peak factor to use in the calculation, by default 2.3.

\end{itemize}

\sphinxlineitem{Returns}
\sphinxAtStartPar
The graph with updated node attributes for peak flow, average daily flow, and upstream pe.

\sphinxlineitem{Return type}
\sphinxAtStartPar
networkx.Graph

\end{description}\end{quote}

\end{fulllineitems}

\index{get\_downstream\_junction() (in module pysewer.optimization)@\spxentry{get\_downstream\_junction()}\spxextra{in module pysewer.optimization}}

\begin{fulllineitems}
\phantomsection\label{\detokenize{pysewer:pysewer.optimization.get_downstream_junction}}
\pysigstartsignatures
\pysiglinewithargsret{\sphinxcode{\sphinxupquote{pysewer.optimization.}}\sphinxbfcode{\sphinxupquote{get\_downstream\_junction}}}{\sphinxparam{\DUrole{n}{G}\DUrole{p}{:}\DUrole{w}{ }\DUrole{n}{Graph}}\sphinxparamcomma \sphinxparam{\DUrole{n}{node}\DUrole{p}{:}\DUrole{w}{ }\DUrole{n}{int}}}{}
\pysigstopsignatures
\sphinxAtStartPar
Returns the next downstream junction from the specified node in G.
\begin{quote}\begin{description}
\sphinxlineitem{Parameters}\begin{itemize}
\item {} 
\sphinxAtStartPar
\sphinxstyleliteralstrong{\sphinxupquote{G}} (\sphinxstyleliteralemphasis{\sphinxupquote{networkx.Graph}}) \textendash{} The graph to search for the downstream junction.

\item {} 
\sphinxAtStartPar
\sphinxstyleliteralstrong{\sphinxupquote{node}} (\sphinxstyleliteralemphasis{\sphinxupquote{int}}) \textendash{} The node to start the search from.

\end{itemize}

\sphinxlineitem{Returns}
\sphinxAtStartPar
The downstream junction from the specified node in G.

\sphinxlineitem{Return type}
\sphinxAtStartPar
int

\end{description}\end{quote}
\subsubsection*{Notes}

\sphinxAtStartPar
The downstream junction is defined as the next junction in the graph that has a degree greater than 2 or an out\sphinxhyphen{}degree of 0.

\end{fulllineitems}

\index{get\_junction\_front() (in module pysewer.optimization)@\spxentry{get\_junction\_front()}\spxextra{in module pysewer.optimization}}

\begin{fulllineitems}
\phantomsection\label{\detokenize{pysewer:pysewer.optimization.get_junction_front}}
\pysigstartsignatures
\pysiglinewithargsret{\sphinxcode{\sphinxupquote{pysewer.optimization.}}\sphinxbfcode{\sphinxupquote{get\_junction\_front}}}{\sphinxparam{\DUrole{n}{G}\DUrole{p}{:}\DUrole{w}{ }\DUrole{n}{Graph}}\sphinxparamcomma \sphinxparam{\DUrole{n}{junctions}}}{}
\pysigstopsignatures
\sphinxAtStartPar
Returns a list of junctions or terminals which have as many entries for inflow trench depths as they have incoming edges.
\begin{quote}\begin{description}
\sphinxlineitem{Parameters}\begin{itemize}
\item {} 
\sphinxAtStartPar
\sphinxstyleliteralstrong{\sphinxupquote{G}} (\sphinxstyleliteralemphasis{\sphinxupquote{networkx.DiGraph}}) \textendash{} A directed graph representing the sewer network.

\item {} 
\sphinxAtStartPar
\sphinxstyleliteralstrong{\sphinxupquote{junctions}} (\sphinxstyleliteralemphasis{\sphinxupquote{list}}) \textendash{} A list of junctions or terminals in the sewer network.

\end{itemize}

\sphinxlineitem{Returns}
\sphinxAtStartPar
A list of junctions or terminals which have as many entries for inflow trench depths as they have incoming edges.

\sphinxlineitem{Return type}
\sphinxAtStartPar
list

\end{description}\end{quote}

\end{fulllineitems}

\index{get\_max\_upstream\_diameter() (in module pysewer.optimization)@\spxentry{get\_max\_upstream\_diameter()}\spxextra{in module pysewer.optimization}}

\begin{fulllineitems}
\phantomsection\label{\detokenize{pysewer:pysewer.optimization.get_max_upstream_diameter}}
\pysigstartsignatures
\pysiglinewithargsret{\sphinxcode{\sphinxupquote{pysewer.optimization.}}\sphinxbfcode{\sphinxupquote{get\_max\_upstream\_diameter}}}{\sphinxparam{\DUrole{n}{G}}\sphinxparamcomma \sphinxparam{\DUrole{n}{edge}\DUrole{p}{:}\DUrole{w}{ }\DUrole{n}{tuple}}}{}
\pysigstopsignatures
\sphinxAtStartPar
Returns the maximum diameter of all upstream edges of the given edge in the directed graph G.
\begin{quote}\begin{description}
\sphinxlineitem{Parameters}\begin{itemize}
\item {} 
\sphinxAtStartPar
\sphinxstyleliteralstrong{\sphinxupquote{G}} (\sphinxstyleliteralemphasis{\sphinxupquote{networkx.DiGraph}}) \textendash{} The directed graph.

\item {} 
\sphinxAtStartPar
\sphinxstyleliteralstrong{\sphinxupquote{edge}} (\sphinxstyleliteralemphasis{\sphinxupquote{tuple}}) \textendash{} The edge for which to find the maximum upstream diameter.

\end{itemize}

\sphinxlineitem{Returns}
\sphinxAtStartPar
The maximum diameter of all upstream edges of the given edge.

\sphinxlineitem{Return type}
\sphinxAtStartPar
float

\end{description}\end{quote}

\end{fulllineitems}

\index{mannings\_equation() (in module pysewer.optimization)@\spxentry{mannings\_equation()}\spxextra{in module pysewer.optimization}}

\begin{fulllineitems}
\phantomsection\label{\detokenize{pysewer:pysewer.optimization.mannings_equation}}
\pysigstartsignatures
\pysiglinewithargsret{\sphinxcode{\sphinxupquote{pysewer.optimization.}}\sphinxbfcode{\sphinxupquote{mannings\_equation}}}{\sphinxparam{\DUrole{n}{pipe\_diameter}\DUrole{p}{:}\DUrole{w}{ }\DUrole{n}{float}}\sphinxparamcomma \sphinxparam{\DUrole{n}{roughness}\DUrole{p}{:}\DUrole{w}{ }\DUrole{n}{float}}\sphinxparamcomma \sphinxparam{\DUrole{n}{slope}\DUrole{p}{:}\DUrole{w}{ }\DUrole{n}{float}}}{{ $\rightarrow$ float}}
\pysigstopsignatures
\sphinxAtStartPar
Calculates the volume flow rate of a pipe using Manning’s equation.
\begin{quote}\begin{description}
\sphinxlineitem{Parameters}\begin{itemize}
\item {} 
\sphinxAtStartPar
\sphinxstyleliteralstrong{\sphinxupquote{pipe\_diameter}} (\sphinxstyleliteralemphasis{\sphinxupquote{float}}) \textendash{} Diameter of the pipe in meters.

\item {} 
\sphinxAtStartPar
\sphinxstyleliteralstrong{\sphinxupquote{roughness}} (\sphinxstyleliteralemphasis{\sphinxupquote{float}}) \textendash{} Roughness coefficient of the pipe.

\item {} 
\sphinxAtStartPar
\sphinxstyleliteralstrong{\sphinxupquote{slope}} (\sphinxstyleliteralemphasis{\sphinxupquote{float}}) \textendash{} Slope of the pipe in units of elevation drop per unit length.

\end{itemize}

\sphinxlineitem{Returns}
\sphinxAtStartPar
Volume flow rate of the pipe in cubic meters per second.

\sphinxlineitem{Return type}
\sphinxAtStartPar
float

\sphinxlineitem{Raises}
\sphinxAtStartPar
\sphinxstyleliteralstrong{\sphinxupquote{ValueError}} \textendash{} If the slope is greater than 0.

\end{description}\end{quote}
\subsubsection*{Notes}

\sphinxAtStartPar
Manning’s equation is used to calculate the volume flow rate of a pipe based on its diameter, roughness coefficient, and slope.

\end{fulllineitems}

\index{needs\_pump() (in module pysewer.optimization)@\spxentry{needs\_pump()}\spxextra{in module pysewer.optimization}}

\begin{fulllineitems}
\phantomsection\label{\detokenize{pysewer:pysewer.optimization.needs_pump}}
\pysigstartsignatures
\pysiglinewithargsret{\sphinxcode{\sphinxupquote{pysewer.optimization.}}\sphinxbfcode{\sphinxupquote{needs\_pump}}}{\sphinxparam{\DUrole{n}{profile}}\sphinxparamcomma \sphinxparam{\DUrole{n}{min\_slope}\DUrole{p}{:}\DUrole{w}{ }\DUrole{n}{float}\DUrole{w}{ }\DUrole{o}{=}\DUrole{w}{ }\DUrole{default_value}{\sphinxhyphen{}0.01}}\sphinxparamcomma \sphinxparam{\DUrole{n}{tmax}\DUrole{p}{:}\DUrole{w}{ }\DUrole{n}{float}\DUrole{w}{ }\DUrole{o}{=}\DUrole{w}{ }\DUrole{default_value}{8}}\sphinxparamcomma \sphinxparam{\DUrole{n}{tmin}\DUrole{p}{:}\DUrole{w}{ }\DUrole{n}{float}\DUrole{w}{ }\DUrole{o}{=}\DUrole{w}{ }\DUrole{default_value}{0.25}}\sphinxparamcomma \sphinxparam{\DUrole{n}{inflow\_trench\_depth}\DUrole{p}{:}\DUrole{w}{ }\DUrole{n}{float}\DUrole{w}{ }\DUrole{o}{=}\DUrole{w}{ }\DUrole{default_value}{0}}}{}
\pysigstopsignatures
\sphinxAtStartPar
Traces a profile to determine if gravitational flow can be achieved within slope and trench depth constraints.
\begin{quote}\begin{description}
\sphinxlineitem{Parameters}\begin{itemize}
\item {} 
\sphinxAtStartPar
\sphinxstyleliteralstrong{\sphinxupquote{profile}} (\sphinxstyleliteralemphasis{\sphinxupquote{list}}\sphinxstyleliteralemphasis{\sphinxupquote{ of }}\sphinxstyleliteralemphasis{\sphinxupquote{tuples}}) \textendash{} A list of (x, y) tuples representing the profile to be traced.

\item {} 
\sphinxAtStartPar
\sphinxstyleliteralstrong{\sphinxupquote{min\_slope}} (\sphinxstyleliteralemphasis{\sphinxupquote{float}}\sphinxstyleliteralemphasis{\sphinxupquote{, }}\sphinxstyleliteralemphasis{\sphinxupquote{optional}}) \textendash{} The minimum slope required for gravitational flow. Default is \sphinxhyphen{}0.01.

\item {} 
\sphinxAtStartPar
\sphinxstyleliteralstrong{\sphinxupquote{tmax}} (\sphinxstyleliteralemphasis{\sphinxupquote{float}}\sphinxstyleliteralemphasis{\sphinxupquote{, }}\sphinxstyleliteralemphasis{\sphinxupquote{optional}}) \textendash{} The maximum trench depth allowed. Default is 8.

\item {} 
\sphinxAtStartPar
\sphinxstyleliteralstrong{\sphinxupquote{tmin}} (\sphinxstyleliteralemphasis{\sphinxupquote{float}}\sphinxstyleliteralemphasis{\sphinxupquote{, }}\sphinxstyleliteralemphasis{\sphinxupquote{optional}}) \textendash{} The minimum trench depth allowed. Default is 0.25.

\item {} 
\sphinxAtStartPar
\sphinxstyleliteralstrong{\sphinxupquote{inflow\_trench\_depth}} (\sphinxstyleliteralemphasis{\sphinxupquote{float}}\sphinxstyleliteralemphasis{\sphinxupquote{, }}\sphinxstyleliteralemphasis{\sphinxupquote{optional}}) \textendash{} The trench depth at the inflow point. If not specified, it is set to tmin.

\end{itemize}

\sphinxlineitem{Returns}
\sphinxAtStartPar
A tuple containing:
\sphinxhyphen{} A boolean indicating whether a pump is needed.
\sphinxhyphen{} The height difference between the outflow and the trench depth at the outflow point.
\sphinxhyphen{} A list of (x, trench\_depth) tuples representing the trench depth at each point along the profile.

\sphinxlineitem{Return type}
\sphinxAtStartPar
tuple

\end{description}\end{quote}

\end{fulllineitems}

\index{place\_lifting\_station() (in module pysewer.optimization)@\spxentry{place\_lifting\_station()}\spxextra{in module pysewer.optimization}}

\begin{fulllineitems}
\phantomsection\label{\detokenize{pysewer:pysewer.optimization.place_lifting_station}}
\pysigstartsignatures
\pysiglinewithargsret{\sphinxcode{\sphinxupquote{pysewer.optimization.}}\sphinxbfcode{\sphinxupquote{place\_lifting\_station}}}{\sphinxparam{\DUrole{n}{G}}\sphinxparamcomma \sphinxparam{\DUrole{n}{node}}}{}
\pysigstopsignatures
\sphinxAtStartPar
Places a lifting station at the specified node in the graph.
\begin{quote}\begin{description}
\sphinxlineitem{Parameters}\begin{itemize}
\item {} 
\sphinxAtStartPar
\sphinxstyleliteralstrong{\sphinxupquote{G}} (\sphinxstyleliteralemphasis{\sphinxupquote{networkx.Graph}}) \textendash{} The graph to add the lifting station to.

\item {} 
\sphinxAtStartPar
\sphinxstyleliteralstrong{\sphinxupquote{node}} (\sphinxstyleliteralemphasis{\sphinxupquote{int}}) \textendash{} The node to add the lifting station to.

\end{itemize}

\sphinxlineitem{Returns}
\sphinxAtStartPar
The graph with the added lifting station.

\sphinxlineitem{Return type}
\sphinxAtStartPar
networkx.Graph

\end{description}\end{quote}

\end{fulllineitems}

\index{place\_pump() (in module pysewer.optimization)@\spxentry{place\_pump()}\spxextra{in module pysewer.optimization}}

\begin{fulllineitems}
\phantomsection\label{\detokenize{pysewer:pysewer.optimization.place_pump}}
\pysigstartsignatures
\pysiglinewithargsret{\sphinxcode{\sphinxupquote{pysewer.optimization.}}\sphinxbfcode{\sphinxupquote{place\_pump}}}{\sphinxparam{\DUrole{n}{G}}\sphinxparamcomma \sphinxparam{\DUrole{n}{node}}}{}
\pysigstopsignatures
\sphinxAtStartPar
Places a pump at the specified node in the graph G and sets downstream edges “pressurized” attribute.
\begin{quote}\begin{description}
\sphinxlineitem{Parameters}\begin{itemize}
\item {} 
\sphinxAtStartPar
\sphinxstyleliteralstrong{\sphinxupquote{G}} (\sphinxstyleliteralemphasis{\sphinxupquote{networkx.Graph}}) \textendash{} The graph in which the pump is to be placed.

\item {} 
\sphinxAtStartPar
\sphinxstyleliteralstrong{\sphinxupquote{node}} (\sphinxstyleliteralemphasis{\sphinxupquote{hashable}}) \textendash{} The node at which the pump is to be placed.

\end{itemize}

\sphinxlineitem{Returns}
\sphinxAtStartPar
The graph with the pump placed at the specified node and downstream edges “pressurized” attribute set.

\sphinxlineitem{Return type}
\sphinxAtStartPar
networkx.Graph

\end{description}\end{quote}

\end{fulllineitems}

\index{reverse\_bfs() (in module pysewer.optimization)@\spxentry{reverse\_bfs()}\spxextra{in module pysewer.optimization}}

\begin{fulllineitems}
\phantomsection\label{\detokenize{pysewer:pysewer.optimization.reverse_bfs}}
\pysigstartsignatures
\pysiglinewithargsret{\sphinxcode{\sphinxupquote{pysewer.optimization.}}\sphinxbfcode{\sphinxupquote{reverse\_bfs}}}{\sphinxparam{\DUrole{n}{G}}\sphinxparamcomma \sphinxparam{\DUrole{n}{sink}\DUrole{p}{:}\DUrole{w}{ }\DUrole{n}{str}}\sphinxparamcomma \sphinxparam{\DUrole{n}{include\_private\_sewer}\DUrole{p}{:}\DUrole{w}{ }\DUrole{n}{bool}\DUrole{w}{ }\DUrole{o}{=}\DUrole{w}{ }\DUrole{default_value}{True}}}{}
\pysigstopsignatures
\sphinxAtStartPar
Returns an iterator over edges in a sequential fashion, starting at the terminals (i.e. buildings) and returning all upstream edges of a junction before moving downstream
\begin{quote}\begin{description}
\sphinxlineitem{Parameters}\begin{itemize}
\item {} 
\sphinxAtStartPar
\sphinxstyleliteralstrong{\sphinxupquote{G}} (\sphinxstyleliteralemphasis{\sphinxupquote{networkx.DiGraph}}) \textendash{} The graph to traverse

\item {} 
\sphinxAtStartPar
\sphinxstyleliteralstrong{\sphinxupquote{sink}} (\sphinxstyleliteralemphasis{\sphinxupquote{str}}) \textendash{} The node to start the traversal from

\item {} 
\sphinxAtStartPar
\sphinxstyleliteralstrong{\sphinxupquote{include\_private\_sewer}} (\sphinxstyleliteralemphasis{\sphinxupquote{bool}}\sphinxstyleliteralemphasis{\sphinxupquote{, }}\sphinxstyleliteralemphasis{\sphinxupquote{optional}}) \textendash{} Whether to include private sewer nodes in the traversal, by default True

\end{itemize}

\sphinxlineitem{Yields}
\sphinxAtStartPar
\sphinxstyleemphasis{tuple} \textendash{} A tuple representing an edge in the graph, in the form (source, target)

\end{description}\end{quote}

\end{fulllineitems}

\index{select\_diameter() (in module pysewer.optimization)@\spxentry{select\_diameter()}\spxextra{in module pysewer.optimization}}

\begin{fulllineitems}
\phantomsection\label{\detokenize{pysewer:pysewer.optimization.select_diameter}}
\pysigstartsignatures
\pysiglinewithargsret{\sphinxcode{\sphinxupquote{pysewer.optimization.}}\sphinxbfcode{\sphinxupquote{select\_diameter}}}{\sphinxparam{\DUrole{n}{target\_flow}\DUrole{p}{:}\DUrole{w}{ }\DUrole{n}{float}}\sphinxparamcomma \sphinxparam{\DUrole{n}{diameters}\DUrole{p}{:}\DUrole{w}{ }\DUrole{n}{List\DUrole{p}{{[}}float\DUrole{p}{{]}}}}\sphinxparamcomma \sphinxparam{\DUrole{n}{roughness}\DUrole{p}{:}\DUrole{w}{ }\DUrole{n}{float}}\sphinxparamcomma \sphinxparam{\DUrole{n}{slope}\DUrole{p}{:}\DUrole{w}{ }\DUrole{n}{float}}}{}
\pysigstopsignatures
\sphinxAtStartPar
Returns the minimum pipe diameter.
\begin{quote}\begin{description}
\sphinxlineitem{Parameters}\begin{itemize}
\item {} 
\sphinxAtStartPar
\sphinxstyleliteralstrong{\sphinxupquote{target\_flow}} (\sphinxstyleliteralemphasis{\sphinxupquote{float}}) \textendash{} The target flow rate in cubic meters per second.

\item {} 
\sphinxAtStartPar
\sphinxstyleliteralstrong{\sphinxupquote{diameters}} (\sphinxstyleliteralemphasis{\sphinxupquote{list}}) \textendash{} A list of possible pipe diameters in meters.

\item {} 
\sphinxAtStartPar
\sphinxstyleliteralstrong{\sphinxupquote{roughness}} (\sphinxstyleliteralemphasis{\sphinxupquote{float}}) \textendash{} The pipe roughness coefficient in meters.

\item {} 
\sphinxAtStartPar
\sphinxstyleliteralstrong{\sphinxupquote{slope}} (\sphinxstyleliteralemphasis{\sphinxupquote{float}}) \textendash{} The pipe slope in meters per meter.

\end{itemize}

\sphinxlineitem{Returns}
\sphinxAtStartPar
The minimum pipe diameter required to achieve the target flow rate.

\sphinxlineitem{Return type}
\sphinxAtStartPar
float

\sphinxlineitem{Raises}
\sphinxAtStartPar
\sphinxstyleliteralstrong{\sphinxupquote{ValueError}} \textendash{} If the maximum diameter is insufficient to reach the target flow rate.

\end{description}\end{quote}

\end{fulllineitems}

\index{set\_diameter() (in module pysewer.optimization)@\spxentry{set\_diameter()}\spxextra{in module pysewer.optimization}}

\begin{fulllineitems}
\phantomsection\label{\detokenize{pysewer:pysewer.optimization.set_diameter}}
\pysigstartsignatures
\pysiglinewithargsret{\sphinxcode{\sphinxupquote{pysewer.optimization.}}\sphinxbfcode{\sphinxupquote{set\_diameter}}}{\sphinxparam{\DUrole{n}{G}\DUrole{p}{:}\DUrole{w}{ }\DUrole{n}{Graph}}\sphinxparamcomma \sphinxparam{\DUrole{n}{edge}\DUrole{p}{:}\DUrole{w}{ }\DUrole{n}{tuple}}\sphinxparamcomma \sphinxparam{\DUrole{n}{diameter}\DUrole{p}{:}\DUrole{w}{ }\DUrole{n}{float}}}{}
\pysigstopsignatures
\sphinxAtStartPar
Set the diameter of an edge in a graph.


\subsection{Parameters:}
\label{\detokenize{pysewer:id9}}\begin{description}
\sphinxlineitem{G}{[}networkx.Graph{]}
\sphinxAtStartPar
The graph to modify.

\sphinxlineitem{edge}{[}tuple{]}
\sphinxAtStartPar
The edge to modify.

\sphinxlineitem{diameter}{[}float{]}
\sphinxAtStartPar
The diameter to set.

\end{description}


\subsection{Returns:}
\label{\detokenize{pysewer:id10}}\begin{description}
\sphinxlineitem{networkx.Graph}
\sphinxAtStartPar
The modified graph.

\end{description}

\end{fulllineitems}



\section{Visualization}
\label{\detokenize{pysewer:module-pysewer.plotting}}\label{\detokenize{pysewer:visualization}}\index{module@\spxentry{module}!pysewer.plotting@\spxentry{pysewer.plotting}}\index{pysewer.plotting@\spxentry{pysewer.plotting}!module@\spxentry{module}}\index{plot\_model\_domain() (in module pysewer.plotting)@\spxentry{plot\_model\_domain()}\spxextra{in module pysewer.plotting}}

\begin{fulllineitems}
\phantomsection\label{\detokenize{pysewer:pysewer.plotting.plot_model_domain}}
\pysigstartsignatures
\pysiglinewithargsret{\sphinxcode{\sphinxupquote{pysewer.plotting.}}\sphinxbfcode{\sphinxupquote{plot\_model\_domain}}}{\sphinxparam{\DUrole{n}{modelDomain}}\sphinxparamcomma \sphinxparam{\DUrole{n}{plot\_connection\_graph}\DUrole{p}{:}\DUrole{w}{ }\DUrole{n}{bool}\DUrole{w}{ }\DUrole{o}{=}\DUrole{w}{ }\DUrole{default_value}{False}}\sphinxparamcomma \sphinxparam{\DUrole{n}{plot\_junction\_graph}\DUrole{p}{:}\DUrole{w}{ }\DUrole{n}{bool}\DUrole{w}{ }\DUrole{o}{=}\DUrole{w}{ }\DUrole{default_value}{False}}\sphinxparamcomma \sphinxparam{\DUrole{n}{plot\_sink}\DUrole{p}{:}\DUrole{w}{ }\DUrole{n}{bool}\DUrole{w}{ }\DUrole{o}{=}\DUrole{w}{ }\DUrole{default_value}{False}}\sphinxparamcomma \sphinxparam{\DUrole{n}{plot\_sewer}\DUrole{p}{:}\DUrole{w}{ }\DUrole{n}{bool}\DUrole{w}{ }\DUrole{o}{=}\DUrole{w}{ }\DUrole{default_value}{False}}\sphinxparamcomma \sphinxparam{\DUrole{n}{sewer\_graph}\DUrole{p}{:}\DUrole{w}{ }\DUrole{n}{Graph\DUrole{w}{ }\DUrole{p}{|}\DUrole{w}{ }None}\DUrole{w}{ }\DUrole{o}{=}\DUrole{w}{ }\DUrole{default_value}{None}}\sphinxparamcomma \sphinxparam{\DUrole{n}{info\_table}\DUrole{p}{:}\DUrole{w}{ }\DUrole{n}{dict\DUrole{w}{ }\DUrole{p}{|}\DUrole{w}{ }None}\DUrole{w}{ }\DUrole{o}{=}\DUrole{w}{ }\DUrole{default_value}{None}}\sphinxparamcomma \sphinxparam{\DUrole{n}{hs\_alt}\DUrole{o}{=}\DUrole{default_value}{30}}\sphinxparamcomma \sphinxparam{\DUrole{n}{hs\_az}\DUrole{o}{=}\DUrole{default_value}{0}}\sphinxparamcomma \sphinxparam{\DUrole{n}{hillshade}\DUrole{p}{:}\DUrole{w}{ }\DUrole{n}{bool}\DUrole{w}{ }\DUrole{o}{=}\DUrole{w}{ }\DUrole{default_value}{False}}\sphinxparamcomma \sphinxparam{\DUrole{n}{fig\_size}\DUrole{p}{:}\DUrole{w}{ }\DUrole{n}{tuple}\DUrole{w}{ }\DUrole{o}{=}\DUrole{w}{ }\DUrole{default_value}{(20, 20)}}}{}
\pysigstopsignatures
\sphinxAtStartPar
Plots the sewer network model domain.
\begin{quote}\begin{description}
\sphinxlineitem{Parameters}\begin{itemize}
\item {} 
\sphinxAtStartPar
\sphinxstyleliteralstrong{\sphinxupquote{modelDomain}} (\sphinxstyleliteralemphasis{\sphinxupquote{pysewer.ModelDomain}}) \textendash{} The model domain to plot.

\item {} 
\sphinxAtStartPar
\sphinxstyleliteralstrong{\sphinxupquote{plot\_connection\_graph}} (\sphinxstyleliteralemphasis{\sphinxupquote{bool}}\sphinxstyleliteralemphasis{\sphinxupquote{, }}\sphinxstyleliteralemphasis{\sphinxupquote{optional}}) \textendash{} Whether to plot the connection graph, by default False.

\item {} 
\sphinxAtStartPar
\sphinxstyleliteralstrong{\sphinxupquote{plot\_junction\_graph}} (\sphinxstyleliteralemphasis{\sphinxupquote{bool}}\sphinxstyleliteralemphasis{\sphinxupquote{, }}\sphinxstyleliteralemphasis{\sphinxupquote{optional}}) \textendash{} Whether to plot the junction graph, by default False.

\item {} 
\sphinxAtStartPar
\sphinxstyleliteralstrong{\sphinxupquote{plot\_sink}} (\sphinxstyleliteralemphasis{\sphinxupquote{bool}}\sphinxstyleliteralemphasis{\sphinxupquote{, }}\sphinxstyleliteralemphasis{\sphinxupquote{optional}}) \textendash{} Whether to plot the sink, by default True.

\item {} 
\sphinxAtStartPar
\sphinxstyleliteralstrong{\sphinxupquote{plot\_sewer}} (\sphinxstyleliteralemphasis{\sphinxupquote{bool}}\sphinxstyleliteralemphasis{\sphinxupquote{, }}\sphinxstyleliteralemphasis{\sphinxupquote{optional}}) \textendash{} Whether to plot the sewer, by default False.

\item {} 
\sphinxAtStartPar
\sphinxstyleliteralstrong{\sphinxupquote{sewer\_graph}} (\sphinxstyleliteralemphasis{\sphinxupquote{networkx.Graph}}\sphinxstyleliteralemphasis{\sphinxupquote{, }}\sphinxstyleliteralemphasis{\sphinxupquote{optional}}) \textendash{} The sewer graph to plot, by default None.

\item {} 
\sphinxAtStartPar
\sphinxstyleliteralstrong{\sphinxupquote{info\_table}} (\sphinxstyleliteralemphasis{\sphinxupquote{dict}}\sphinxstyleliteralemphasis{\sphinxupquote{, }}\sphinxstyleliteralemphasis{\sphinxupquote{optional}}) \textendash{} The information table to plot, by default None.

\item {} 
\sphinxAtStartPar
\sphinxstyleliteralstrong{\sphinxupquote{hs\_alt}} (\sphinxstyleliteralemphasis{\sphinxupquote{int}}\sphinxstyleliteralemphasis{\sphinxupquote{, }}\sphinxstyleliteralemphasis{\sphinxupquote{optional}}) \textendash{} The altitude of the hillshade, by default 30.

\item {} 
\sphinxAtStartPar
\sphinxstyleliteralstrong{\sphinxupquote{hs\_az}} (\sphinxstyleliteralemphasis{\sphinxupquote{int}}\sphinxstyleliteralemphasis{\sphinxupquote{, }}\sphinxstyleliteralemphasis{\sphinxupquote{optional}}) \textendash{} The azimuth of the hillshade, by default 0.

\item {} 
\sphinxAtStartPar
\sphinxstyleliteralstrong{\sphinxupquote{hillshade}} (\sphinxstyleliteralemphasis{\sphinxupquote{bool}}\sphinxstyleliteralemphasis{\sphinxupquote{, }}\sphinxstyleliteralemphasis{\sphinxupquote{optional}}) \textendash{} Whether to plot the hillshade, by default False.

\end{itemize}

\sphinxlineitem{Returns}
\sphinxAtStartPar
\sphinxstylestrong{fig, ax} \textendash{} The figure and axes of the plot.

\sphinxlineitem{Return type}
\sphinxAtStartPar
matplotlib.figure.Figure, matplotlib.axes.Axes

\end{description}\end{quote}

\end{fulllineitems}

\index{plot\_sewer\_attributes() (in module pysewer.plotting)@\spxentry{plot\_sewer\_attributes()}\spxextra{in module pysewer.plotting}}

\begin{fulllineitems}
\phantomsection\label{\detokenize{pysewer:pysewer.plotting.plot_sewer_attributes}}
\pysigstartsignatures
\pysiglinewithargsret{\sphinxcode{\sphinxupquote{pysewer.plotting.}}\sphinxbfcode{\sphinxupquote{plot\_sewer\_attributes}}}{\sphinxparam{\DUrole{n}{modelDomain}}\sphinxparamcomma \sphinxparam{\DUrole{n}{sewer\_graph}}\sphinxparamcomma \sphinxparam{\DUrole{n}{attribute}}\sphinxparamcomma \sphinxparam{\DUrole{n}{colormap}\DUrole{o}{=}\DUrole{default_value}{\textquotesingle{}jet\textquotesingle{}}}\sphinxparamcomma \sphinxparam{\DUrole{n}{title}\DUrole{o}{=}\DUrole{default_value}{\textquotesingle{}Sewer Network Plot\textquotesingle{}}}\sphinxparamcomma \sphinxparam{\DUrole{n}{hillshade}\DUrole{o}{=}\DUrole{default_value}{False}}\sphinxparamcomma \sphinxparam{\DUrole{n}{fig\_size}\DUrole{o}{=}\DUrole{default_value}{(20, 20)}}}{}
\pysigstopsignatures
\sphinxAtStartPar
Plots the sewer network with the specified attribute.
\begin{quote}\begin{description}
\sphinxlineitem{Parameters}\begin{itemize}
\item {} 
\sphinxAtStartPar
\sphinxstyleliteralstrong{\sphinxupquote{modelDomain}} (\sphinxstyleliteralemphasis{\sphinxupquote{object}}) \textendash{} The model domain object.

\item {} 
\sphinxAtStartPar
\sphinxstyleliteralstrong{\sphinxupquote{sewer\_graph}} (\sphinxstyleliteralemphasis{\sphinxupquote{object}}) \textendash{} The sewer graph object.

\item {} 
\sphinxAtStartPar
\sphinxstyleliteralstrong{\sphinxupquote{attribute}} (\sphinxstyleliteralemphasis{\sphinxupquote{str}}) \textendash{} The attribute to plot.

\item {} 
\sphinxAtStartPar
\sphinxstyleliteralstrong{\sphinxupquote{colormap}} (\sphinxstyleliteralemphasis{\sphinxupquote{str}}\sphinxstyleliteralemphasis{\sphinxupquote{, }}\sphinxstyleliteralemphasis{\sphinxupquote{optional}}) \textendash{} The colormap to use for the plot. Default is “jet”.

\item {} 
\sphinxAtStartPar
\sphinxstyleliteralstrong{\sphinxupquote{title}} (\sphinxstyleliteralemphasis{\sphinxupquote{str}}\sphinxstyleliteralemphasis{\sphinxupquote{, }}\sphinxstyleliteralemphasis{\sphinxupquote{optional}}) \textendash{} The title of the plot. Default is “Sewer Network Plot”.

\item {} 
\sphinxAtStartPar
\sphinxstyleliteralstrong{\sphinxupquote{hillshade}} (\sphinxstyleliteralemphasis{\sphinxupquote{bool}}\sphinxstyleliteralemphasis{\sphinxupquote{, }}\sphinxstyleliteralemphasis{\sphinxupquote{optional}}) \textendash{} Whether to include a hillshade in the plot. Default is False.

\end{itemize}

\sphinxlineitem{Returns}
\sphinxAtStartPar
\begin{itemize}
\item {} 
\sphinxAtStartPar
\sphinxstylestrong{fig} (\sphinxstyleemphasis{matplotlib.figure.Figure}) \textendash{} The figure object.

\item {} 
\sphinxAtStartPar
\sphinxstylestrong{ax} (\sphinxstyleemphasis{matplotlib.axes.Axes}) \textendash{} The axes object.

\end{itemize}


\end{description}\end{quote}

\end{fulllineitems}



\section{Preprocessing}
\label{\detokenize{pysewer:module-pysewer.preprocessing}}\label{\detokenize{pysewer:preprocessing}}\index{module@\spxentry{module}!pysewer.preprocessing@\spxentry{pysewer.preprocessing}}\index{pysewer.preprocessing@\spxentry{pysewer.preprocessing}!module@\spxentry{module}}\index{Buildings (class in pysewer.preprocessing)@\spxentry{Buildings}\spxextra{class in pysewer.preprocessing}}

\begin{fulllineitems}
\phantomsection\label{\detokenize{pysewer:pysewer.preprocessing.Buildings}}
\pysigstartsignatures
\pysiglinewithargsret{\sphinxbfcode{\sphinxupquote{class\DUrole{w}{ }}}\sphinxcode{\sphinxupquote{pysewer.preprocessing.}}\sphinxbfcode{\sphinxupquote{Buildings}}}{\sphinxparam{\DUrole{n}{input\_data}\DUrole{p}{:}\DUrole{w}{ }\DUrole{n}{str\DUrole{w}{ }\DUrole{p}{|}\DUrole{w}{ }GeoDataFrame}}\sphinxparamcomma \sphinxparam{\DUrole{n}{roads\_obj}\DUrole{p}{:}\DUrole{w}{ }\DUrole{n}{{\hyperref[\detokenize{pysewer:pysewer.preprocessing.Roads}]{\sphinxcrossref{Roads}}}}}}{}
\pysigstopsignatures
\sphinxAtStartPar
Bases: \sphinxcode{\sphinxupquote{object}}

\sphinxAtStartPar
A class to preprocess building data.
\begin{quote}\begin{description}
\sphinxlineitem{Parameters}\begin{itemize}
\item {} 
\sphinxAtStartPar
\sphinxstyleliteralstrong{\sphinxupquote{input\_data}} (\sphinxstyleliteralemphasis{\sphinxupquote{str}}\sphinxstyleliteralemphasis{\sphinxupquote{ or }}\sphinxstyleliteralemphasis{\sphinxupquote{geopandas.GeoDataFrame}}) \textendash{} Path to a shapefile or a GeoDataFrame containing the input data.

\item {} 
\sphinxAtStartPar
\sphinxstyleliteralstrong{\sphinxupquote{roads\_obj}} (\sphinxstyleliteralemphasis{\sphinxupquote{pysewer.Roads}}) \textendash{} A Roads object containing the road network data.

\end{itemize}

\end{description}\end{quote}
\index{gdf (pysewer.preprocessing.Buildings attribute)@\spxentry{gdf}\spxextra{pysewer.preprocessing.Buildings attribute}}

\begin{fulllineitems}
\phantomsection\label{\detokenize{pysewer:pysewer.preprocessing.Buildings.gdf}}
\pysigstartsignatures
\pysigline{\sphinxbfcode{\sphinxupquote{gdf}}}
\pysigstopsignatures
\sphinxAtStartPar
The input building data.
\begin{quote}\begin{description}
\sphinxlineitem{Type}
\sphinxAtStartPar
geopandas.GeoDataFrame

\end{description}\end{quote}

\end{fulllineitems}

\index{roads\_obj (pysewer.preprocessing.Buildings attribute)@\spxentry{roads\_obj}\spxextra{pysewer.preprocessing.Buildings attribute}}

\begin{fulllineitems}
\phantomsection\label{\detokenize{pysewer:pysewer.preprocessing.Buildings.roads_obj}}
\pysigstartsignatures
\pysigline{\sphinxbfcode{\sphinxupquote{roads\_obj}}}
\pysigstopsignatures
\sphinxAtStartPar
The road network data.
\begin{quote}\begin{description}
\sphinxlineitem{Type}
\sphinxAtStartPar
pysewer.Roads

\end{description}\end{quote}

\end{fulllineitems}



\begin{fulllineitems}

\pysigstartsignatures
\pysigline{\sphinxbfcode{\sphinxupquote{get\_gdf():}}}
\pysigstopsignatures
\sphinxAtStartPar
Returns building data as geopandas dataframe.

\end{fulllineitems}



\begin{fulllineitems}

\pysigstartsignatures
\pysigline{\sphinxbfcode{\sphinxupquote{get\_crs():}}}
\pysigstopsignatures
\sphinxAtStartPar
Returns the Coordinate System of the DEM Object.

\end{fulllineitems}



\begin{fulllineitems}

\pysigstartsignatures
\pysigline{\sphinxbfcode{\sphinxupquote{cluster\_centers(max\_connection\_length):}}}
\pysigstopsignatures
\sphinxAtStartPar
Returns a list of cluster centers for all buildings with greater than max\_connection\_length distance to the nearest street.

\end{fulllineitems}

\index{cluster\_centers() (pysewer.preprocessing.Buildings method)@\spxentry{cluster\_centers()}\spxextra{pysewer.preprocessing.Buildings method}}

\begin{fulllineitems}
\phantomsection\label{\detokenize{pysewer:pysewer.preprocessing.Buildings.cluster_centers}}
\pysigstartsignatures
\pysiglinewithargsret{\sphinxbfcode{\sphinxupquote{cluster\_centers}}}{\sphinxparam{\DUrole{n}{max\_connection\_length}\DUrole{p}{:}\DUrole{w}{ }\DUrole{n}{float}}}{}
\pysigstopsignatures
\sphinxAtStartPar
Returns a list of cluster centers for all buildings with greater than max\_connection\_length distance to the nearest street.
\begin{quote}\begin{description}
\sphinxlineitem{Parameters}
\sphinxAtStartPar
\sphinxstyleliteralstrong{\sphinxupquote{max\_connection\_length}} (\sphinxstyleliteralemphasis{\sphinxupquote{float}}) \textendash{} The maximum distance between a building and the nearest street for it to be included in the cluster centers list.

\sphinxlineitem{Returns}
\sphinxAtStartPar
A GeoDataFrame containing the cluster centers and their distances to the nearest street, sorted by distance.

\sphinxlineitem{Return type}
\sphinxAtStartPar
geopandas.GeoDataFrame

\end{description}\end{quote}

\end{fulllineitems}

\index{get\_crs() (pysewer.preprocessing.Buildings method)@\spxentry{get\_crs()}\spxextra{pysewer.preprocessing.Buildings method}}

\begin{fulllineitems}
\phantomsection\label{\detokenize{pysewer:pysewer.preprocessing.Buildings.get_crs}}
\pysigstartsignatures
\pysiglinewithargsret{\sphinxbfcode{\sphinxupquote{get\_crs}}}{}{}
\pysigstopsignatures
\sphinxAtStartPar
Returns the Coordinate System of the DEM Object.
\begin{quote}\begin{description}
\sphinxlineitem{Returns}
\sphinxAtStartPar
The Coordinate Reference System (CRS) of the building data.

\sphinxlineitem{Return type}
\sphinxAtStartPar
dict

\end{description}\end{quote}

\end{fulllineitems}

\index{get\_gdf() (pysewer.preprocessing.Buildings method)@\spxentry{get\_gdf()}\spxextra{pysewer.preprocessing.Buildings method}}

\begin{fulllineitems}
\phantomsection\label{\detokenize{pysewer:pysewer.preprocessing.Buildings.get_gdf}}
\pysigstartsignatures
\pysiglinewithargsret{\sphinxbfcode{\sphinxupquote{get\_gdf}}}{}{}
\pysigstopsignatures
\sphinxAtStartPar
Returns building data as geopandas dataframe.
\begin{quote}\begin{description}
\sphinxlineitem{Returns}
\sphinxAtStartPar
The building data.

\sphinxlineitem{Return type}
\sphinxAtStartPar
geopandas.GeoDataFrame

\end{description}\end{quote}

\end{fulllineitems}


\end{fulllineitems}

\index{DEM (class in pysewer.preprocessing)@\spxentry{DEM}\spxextra{class in pysewer.preprocessing}}

\begin{fulllineitems}
\phantomsection\label{\detokenize{pysewer:pysewer.preprocessing.DEM}}
\pysigstartsignatures
\pysiglinewithargsret{\sphinxbfcode{\sphinxupquote{class\DUrole{w}{ }}}\sphinxcode{\sphinxupquote{pysewer.preprocessing.}}\sphinxbfcode{\sphinxupquote{DEM}}}{\sphinxparam{\DUrole{n}{file\_path}\DUrole{p}{:}\DUrole{w}{ }\DUrole{n}{str\DUrole{w}{ }\DUrole{p}{|}\DUrole{w}{ }None}\DUrole{w}{ }\DUrole{o}{=}\DUrole{w}{ }\DUrole{default_value}{None}}}{}
\pysigstopsignatures
\sphinxAtStartPar
Bases: \sphinxcode{\sphinxupquote{object}}

\sphinxAtStartPar
A class for handling digital elevation model (DEM) data.
\index{file\_path (pysewer.preprocessing.DEM attribute)@\spxentry{file\_path}\spxextra{pysewer.preprocessing.DEM attribute}}

\begin{fulllineitems}
\phantomsection\label{\detokenize{pysewer:pysewer.preprocessing.DEM.file_path}}
\pysigstartsignatures
\pysigline{\sphinxbfcode{\sphinxupquote{file\_path}}}
\pysigstopsignatures
\sphinxAtStartPar
The file path to the DEM raster file.
\begin{quote}\begin{description}
\sphinxlineitem{Type}
\sphinxAtStartPar
str, optional

\end{description}\end{quote}

\end{fulllineitems}

\index{raster (pysewer.preprocessing.DEM attribute)@\spxentry{raster}\spxextra{pysewer.preprocessing.DEM attribute}}

\begin{fulllineitems}
\phantomsection\label{\detokenize{pysewer:pysewer.preprocessing.DEM.raster}}
\pysigstartsignatures
\pysigline{\sphinxbfcode{\sphinxupquote{raster}}}
\pysigstopsignatures
\sphinxAtStartPar
The rasterio dataset reader object for the DEM raster file.
\begin{quote}\begin{description}
\sphinxlineitem{Type}
\sphinxAtStartPar
rasterio.DatasetReader, optional

\end{description}\end{quote}

\end{fulllineitems}

\index{no\_dem (pysewer.preprocessing.DEM attribute)@\spxentry{no\_dem}\spxextra{pysewer.preprocessing.DEM attribute}}

\begin{fulllineitems}
\phantomsection\label{\detokenize{pysewer:pysewer.preprocessing.DEM.no_dem}}
\pysigstartsignatures
\pysigline{\sphinxbfcode{\sphinxupquote{no\_dem}}}
\pysigstopsignatures
\sphinxAtStartPar
A flag indicating whether or not a DEM has been loaded.
\begin{quote}\begin{description}
\sphinxlineitem{Type}
\sphinxAtStartPar
bool

\end{description}\end{quote}

\end{fulllineitems}

\index{get\_elevation() (pysewer.preprocessing.DEM method)@\spxentry{get\_elevation()}\spxextra{pysewer.preprocessing.DEM method}}

\begin{fulllineitems}
\phantomsection\label{\detokenize{pysewer:pysewer.preprocessing.DEM.get_elevation}}
\pysigstartsignatures
\pysiglinewithargsret{\sphinxbfcode{\sphinxupquote{get\_elevation}}}{\sphinxparam{\DUrole{n}{point}\DUrole{p}{:}\DUrole{w}{ }\DUrole{n}{shapely.geometry.Point}}}{{ $\rightarrow$ int:}}
\pysigstopsignatures
\sphinxAtStartPar
Returns elevation data in meters for a given point rounded to two decimals.

\end{fulllineitems}

\index{get\_profile() (pysewer.preprocessing.DEM method)@\spxentry{get\_profile()}\spxextra{pysewer.preprocessing.DEM method}}

\begin{fulllineitems}
\phantomsection\label{\detokenize{pysewer:pysewer.preprocessing.DEM.get_profile}}
\pysigstartsignatures
\pysiglinewithargsret{\sphinxbfcode{\sphinxupquote{get\_profile}}}{\sphinxparam{\DUrole{n}{line}\DUrole{p}{:}\DUrole{w}{ }\DUrole{n}{shapely.geometry.LineString}}\sphinxparamcomma \sphinxparam{\DUrole{n}{dx}\DUrole{p}{:}\DUrole{w}{ }\DUrole{n}{int}\DUrole{w}{ }\DUrole{o}{=}\DUrole{w}{ }\DUrole{default_value}{10}}}{{ $\rightarrow$ List{[}Tuple{[}float, int{]}{]}:}}
\pysigstopsignatures
\sphinxAtStartPar
Extracts elevation data from a digital elevation model (DEM) along a given path.

\end{fulllineitems}

\index{reproject\_dem() (pysewer.preprocessing.DEM method)@\spxentry{reproject\_dem()}\spxextra{pysewer.preprocessing.DEM method}}

\begin{fulllineitems}
\phantomsection\label{\detokenize{pysewer:pysewer.preprocessing.DEM.reproject_dem}}
\pysigstartsignatures
\pysiglinewithargsret{\sphinxbfcode{\sphinxupquote{reproject\_dem}}}{\sphinxparam{\DUrole{n}{crs}\DUrole{p}{:}\DUrole{w}{ }\DUrole{n}{CRS}}}{{ $\rightarrow$ None:}}
\pysigstopsignatures
\sphinxAtStartPar
Reprojects the DEM raster to the given CRS.

\end{fulllineitems}

\index{Properties() (pysewer.preprocessing.DEM method)@\spxentry{Properties()}\spxextra{pysewer.preprocessing.DEM method}}

\begin{fulllineitems}
\phantomsection\label{\detokenize{pysewer:pysewer.preprocessing.DEM.Properties}}
\pysigstartsignatures
\pysiglinewithargsret{\sphinxbfcode{\sphinxupquote{Properties}}}{}{}
\pysigstopsignatures
\end{fulllineitems}



\begin{fulllineitems}

\pysigstartsignatures
\pysigline{\sphinxbfcode{\sphinxupquote{\sphinxhyphen{}\sphinxhyphen{}\sphinxhyphen{}\sphinxhyphen{}\sphinxhyphen{}\sphinxhyphen{}\sphinxhyphen{}\sphinxhyphen{}\sphinxhyphen{}\sphinxhyphen{}}}}
\pysigstopsignatures
\end{fulllineitems}



\begin{fulllineitems}

\pysigstartsignatures
\pysigline{\sphinxbfcode{\sphinxupquote{get\_crs~:~CRS}}}
\pysigstopsignatures
\sphinxAtStartPar
Returns the coordinate system of the DEM object.

\end{fulllineitems}

\index{file\_path (pysewer.preprocessing.DEM attribute)@\spxentry{file\_path}\spxextra{pysewer.preprocessing.DEM attribute}}

\begin{fulllineitems}
\phantomsection\label{\detokenize{pysewer:id0}}
\pysigstartsignatures
\pysigline{\sphinxbfcode{\sphinxupquote{file\_path}}\sphinxbfcode{\sphinxupquote{\DUrole{p}{:}\DUrole{w}{ }str\DUrole{w}{ }\DUrole{p}{|}\DUrole{w}{ }None}}\sphinxbfcode{\sphinxupquote{\DUrole{w}{ }\DUrole{p}{=}\DUrole{w}{ }None}}}
\pysigstopsignatures
\end{fulllineitems}

\index{get\_crs (pysewer.preprocessing.DEM property)@\spxentry{get\_crs}\spxextra{pysewer.preprocessing.DEM property}}

\begin{fulllineitems}
\phantomsection\label{\detokenize{pysewer:pysewer.preprocessing.DEM.get_crs}}
\pysigstartsignatures
\pysigline{\sphinxbfcode{\sphinxupquote{property\DUrole{w}{ }}}\sphinxbfcode{\sphinxupquote{get\_crs}}\sphinxbfcode{\sphinxupquote{\DUrole{p}{:}\DUrole{w}{ }CRS\DUrole{w}{ }\DUrole{p}{|}\DUrole{w}{ }None}}}
\pysigstopsignatures
\sphinxAtStartPar
Returns the coordinate system of the DEM Object

\end{fulllineitems}

\index{get\_elevation() (pysewer.preprocessing.DEM method)@\spxentry{get\_elevation()}\spxextra{pysewer.preprocessing.DEM method}}

\begin{fulllineitems}
\phantomsection\label{\detokenize{pysewer:id11}}
\pysigstartsignatures
\pysiglinewithargsret{\sphinxbfcode{\sphinxupquote{get\_elevation}}}{\sphinxparam{\DUrole{n}{point}\DUrole{p}{:}\DUrole{w}{ }\DUrole{n}{Point}}}{}
\pysigstopsignatures
\sphinxAtStartPar
Returns elevation data in meters for a given point rounded to two decimals.
\begin{quote}\begin{description}
\sphinxlineitem{Parameters}
\sphinxAtStartPar
\sphinxstyleliteralstrong{\sphinxupquote{point}} (\sphinxstyleliteralemphasis{\sphinxupquote{shapely.geometry.Point}}) \textendash{} The point for which to retrieve elevation data.

\sphinxlineitem{Returns}
\sphinxAtStartPar
The elevation in meters.

\sphinxlineitem{Return type}
\sphinxAtStartPar
int

\sphinxlineitem{Raises}
\sphinxAtStartPar
\sphinxstyleliteralstrong{\sphinxupquote{ValueError}} \textendash{} If the query point is out of bounds or if there is no elevation data for the given coordinates.

\end{description}\end{quote}

\end{fulllineitems}

\index{get\_profile() (pysewer.preprocessing.DEM method)@\spxentry{get\_profile()}\spxextra{pysewer.preprocessing.DEM method}}

\begin{fulllineitems}
\phantomsection\label{\detokenize{pysewer:id12}}
\pysigstartsignatures
\pysiglinewithargsret{\sphinxbfcode{\sphinxupquote{get\_profile}}}{\sphinxparam{\DUrole{n}{line}\DUrole{p}{:}\DUrole{w}{ }\DUrole{n}{LineString}}\sphinxparamcomma \sphinxparam{\DUrole{n}{dx}\DUrole{p}{:}\DUrole{w}{ }\DUrole{n}{int}\DUrole{w}{ }\DUrole{o}{=}\DUrole{w}{ }\DUrole{default_value}{10}}}{}
\pysigstopsignatures
\sphinxAtStartPar
Extracts elevation data from a digital elevation model (DEM) along a given path.
\begin{quote}\begin{description}
\sphinxlineitem{Parameters}\begin{itemize}
\item {} 
\sphinxAtStartPar
\sphinxstyleliteralstrong{\sphinxupquote{line}} (\sphinxstyleliteralemphasis{\sphinxupquote{shapely.geometry.LineString}}) \textendash{} The path along which to extract elevation data.

\item {} 
\sphinxAtStartPar
\sphinxstyleliteralstrong{\sphinxupquote{dx}} (\sphinxstyleliteralemphasis{\sphinxupquote{float}}\sphinxstyleliteralemphasis{\sphinxupquote{, }}\sphinxstyleliteralemphasis{\sphinxupquote{optional}}) \textendash{} The sampling resolution in meters. Default is 10.

\end{itemize}

\sphinxlineitem{Returns}
\sphinxAtStartPar
A list of (x, elevation) tuples representing the x\sphinxhyphen{}coordinate and elevation data of the profile.
The x\sphinxhyphen{}coordinate values start at 0 and are spaced at intervals of dx meters.

\sphinxlineitem{Return type}
\sphinxAtStartPar
List of Tuples

\end{description}\end{quote}

\end{fulllineitems}

\index{no\_dem (pysewer.preprocessing.DEM attribute)@\spxentry{no\_dem}\spxextra{pysewer.preprocessing.DEM attribute}}

\begin{fulllineitems}
\phantomsection\label{\detokenize{pysewer:id13}}
\pysigstartsignatures
\pysigline{\sphinxbfcode{\sphinxupquote{no\_dem}}\sphinxbfcode{\sphinxupquote{\DUrole{p}{:}\DUrole{w}{ }bool}}\sphinxbfcode{\sphinxupquote{\DUrole{w}{ }\DUrole{p}{=}\DUrole{w}{ }True}}}
\pysigstopsignatures
\end{fulllineitems}

\index{raster (pysewer.preprocessing.DEM attribute)@\spxentry{raster}\spxextra{pysewer.preprocessing.DEM attribute}}

\begin{fulllineitems}
\phantomsection\label{\detokenize{pysewer:id14}}
\pysigstartsignatures
\pysigline{\sphinxbfcode{\sphinxupquote{raster}}\sphinxbfcode{\sphinxupquote{\DUrole{p}{:}\DUrole{w}{ }DatasetReader}}\sphinxbfcode{\sphinxupquote{\DUrole{w}{ }\DUrole{p}{=}\DUrole{w}{ }None}}}
\pysigstopsignatures
\end{fulllineitems}

\index{reproject\_dem() (pysewer.preprocessing.DEM method)@\spxentry{reproject\_dem()}\spxextra{pysewer.preprocessing.DEM method}}

\begin{fulllineitems}
\phantomsection\label{\detokenize{pysewer:id15}}
\pysigstartsignatures
\pysiglinewithargsret{\sphinxbfcode{\sphinxupquote{reproject\_dem}}}{\sphinxparam{\DUrole{n}{crs}\DUrole{p}{:}\DUrole{w}{ }\DUrole{n}{CRS}}}{}
\pysigstopsignatures
\sphinxAtStartPar
Reprojects the DEM raster to the given CRS.
\begin{quote}\begin{description}
\sphinxlineitem{Parameters}
\sphinxAtStartPar
\sphinxstyleliteralstrong{\sphinxupquote{crs}} (\sphinxstyleliteralemphasis{\sphinxupquote{CRS}}) \textendash{} The target CRS to reproject the raster to.

\sphinxlineitem{Raises}
\sphinxAtStartPar
\sphinxstyleliteralstrong{\sphinxupquote{ValueError}} \textendash{} If no DEM is loaded, cannot reproject DEM.

\end{description}\end{quote}

\end{fulllineitems}


\end{fulllineitems}

\index{ModelDomain (class in pysewer.preprocessing)@\spxentry{ModelDomain}\spxextra{class in pysewer.preprocessing}}

\begin{fulllineitems}
\phantomsection\label{\detokenize{pysewer:pysewer.preprocessing.ModelDomain}}
\pysigstartsignatures
\pysiglinewithargsret{\sphinxbfcode{\sphinxupquote{class\DUrole{w}{ }}}\sphinxcode{\sphinxupquote{pysewer.preprocessing.}}\sphinxbfcode{\sphinxupquote{ModelDomain}}}{\sphinxparam{\DUrole{n}{dem}\DUrole{p}{:}\DUrole{w}{ }\DUrole{n}{str}}\sphinxparamcomma \sphinxparam{\DUrole{n}{roads}\DUrole{p}{:}\DUrole{w}{ }\DUrole{n}{str}}\sphinxparamcomma \sphinxparam{\DUrole{n}{buildings}\DUrole{p}{:}\DUrole{w}{ }\DUrole{n}{str}}\sphinxparamcomma \sphinxparam{\DUrole{n}{clustering}\DUrole{p}{:}\DUrole{w}{ }\DUrole{n}{str}\DUrole{w}{ }\DUrole{o}{=}\DUrole{w}{ }\DUrole{default_value}{\textquotesingle{}centers\textquotesingle{}}}\sphinxparamcomma \sphinxparam{\DUrole{n}{pump\_penalty}\DUrole{p}{:}\DUrole{w}{ }\DUrole{n}{int}\DUrole{w}{ }\DUrole{o}{=}\DUrole{w}{ }\DUrole{default_value}{1000}}\sphinxparamcomma \sphinxparam{\DUrole{n}{connect\_buildings}\DUrole{p}{:}\DUrole{w}{ }\DUrole{n}{bool}\DUrole{w}{ }\DUrole{o}{=}\DUrole{w}{ }\DUrole{default_value}{True}}}{}
\pysigstopsignatures
\sphinxAtStartPar
Bases: \sphinxcode{\sphinxupquote{object}}

\sphinxAtStartPar
Class for preprocessing input data for the sewer network.
\begin{quote}\begin{description}
\sphinxlineitem{Parameters}\begin{itemize}
\item {} 
\sphinxAtStartPar
\sphinxstyleliteralstrong{\sphinxupquote{dem}} (\sphinxstyleliteralemphasis{\sphinxupquote{str}}) \textendash{} Path to the digital elevation model file.

\item {} 
\sphinxAtStartPar
\sphinxstyleliteralstrong{\sphinxupquote{roads}} (\sphinxstyleliteralemphasis{\sphinxupquote{str}}) \textendash{} Path to the roads shapefile.

\item {} 
\sphinxAtStartPar
\sphinxstyleliteralstrong{\sphinxupquote{buildings}} (\sphinxstyleliteralemphasis{\sphinxupquote{str}}) \textendash{} Path to the buildings shapefile.

\item {} 
\sphinxAtStartPar
\sphinxstyleliteralstrong{\sphinxupquote{clustering}} (\sphinxstyleliteralemphasis{\sphinxupquote{str}}\sphinxstyleliteralemphasis{\sphinxupquote{, }}\sphinxstyleliteralemphasis{\sphinxupquote{optional}}) \textendash{} Clustering method for connecting buildings to the sewer network. Default is “centers”.

\item {} 
\sphinxAtStartPar
\sphinxstyleliteralstrong{\sphinxupquote{pump\_penalty}} (\sphinxstyleliteralemphasis{\sphinxupquote{int}}\sphinxstyleliteralemphasis{\sphinxupquote{, }}\sphinxstyleliteralemphasis{\sphinxupquote{optional}}) \textendash{} Penalty for adding a pump to the sewer network. Default is 1000.

\item {} 
\sphinxAtStartPar
\sphinxstyleliteralstrong{\sphinxupquote{connect\_buildings}} (\sphinxstyleliteralemphasis{\sphinxupquote{bool}}\sphinxstyleliteralemphasis{\sphinxupquote{, }}\sphinxstyleliteralemphasis{\sphinxupquote{optional}}) \textendash{} Whether to connect buildings to the sewer network. Default is True.

\end{itemize}

\end{description}\end{quote}
\index{roads (pysewer.preprocessing.ModelDomain attribute)@\spxentry{roads}\spxextra{pysewer.preprocessing.ModelDomain attribute}}

\begin{fulllineitems}
\phantomsection\label{\detokenize{pysewer:pysewer.preprocessing.ModelDomain.roads}}
\pysigstartsignatures
\pysigline{\sphinxbfcode{\sphinxupquote{roads}}}
\pysigstopsignatures
\sphinxAtStartPar
Roads object.
\begin{quote}\begin{description}
\sphinxlineitem{Type}
\sphinxAtStartPar
{\hyperref[\detokenize{pysewer:pysewer.preprocessing.Roads}]{\sphinxcrossref{Roads}}}

\end{description}\end{quote}

\end{fulllineitems}

\index{buildings (pysewer.preprocessing.ModelDomain attribute)@\spxentry{buildings}\spxextra{pysewer.preprocessing.ModelDomain attribute}}

\begin{fulllineitems}
\phantomsection\label{\detokenize{pysewer:pysewer.preprocessing.ModelDomain.buildings}}
\pysigstartsignatures
\pysigline{\sphinxbfcode{\sphinxupquote{buildings}}}
\pysigstopsignatures
\sphinxAtStartPar
Buildings object.
\begin{quote}\begin{description}
\sphinxlineitem{Type}
\sphinxAtStartPar
{\hyperref[\detokenize{pysewer:pysewer.preprocessing.Buildings}]{\sphinxcrossref{Buildings}}}

\end{description}\end{quote}

\end{fulllineitems}

\index{dem (pysewer.preprocessing.ModelDomain attribute)@\spxentry{dem}\spxextra{pysewer.preprocessing.ModelDomain attribute}}

\begin{fulllineitems}
\phantomsection\label{\detokenize{pysewer:pysewer.preprocessing.ModelDomain.dem}}
\pysigstartsignatures
\pysigline{\sphinxbfcode{\sphinxupquote{dem}}}
\pysigstopsignatures
\sphinxAtStartPar
DEM object.
\begin{quote}\begin{description}
\sphinxlineitem{Type}
\sphinxAtStartPar
{\hyperref[\detokenize{pysewer:pysewer.preprocessing.DEM}]{\sphinxcrossref{DEM}}}

\end{description}\end{quote}

\end{fulllineitems}

\index{connection\_graph (pysewer.preprocessing.ModelDomain attribute)@\spxentry{connection\_graph}\spxextra{pysewer.preprocessing.ModelDomain attribute}}

\begin{fulllineitems}
\phantomsection\label{\detokenize{pysewer:pysewer.preprocessing.ModelDomain.connection_graph}}
\pysigstartsignatures
\pysigline{\sphinxbfcode{\sphinxupquote{connection\_graph}}}
\pysigstopsignatures
\sphinxAtStartPar
Graph representing the road network.
\begin{quote}\begin{description}
\sphinxlineitem{Type}
\sphinxAtStartPar
nx.Graph

\end{description}\end{quote}

\end{fulllineitems}

\index{pump\_penalty (pysewer.preprocessing.ModelDomain attribute)@\spxentry{pump\_penalty}\spxextra{pysewer.preprocessing.ModelDomain attribute}}

\begin{fulllineitems}
\phantomsection\label{\detokenize{pysewer:pysewer.preprocessing.ModelDomain.pump_penalty}}
\pysigstartsignatures
\pysigline{\sphinxbfcode{\sphinxupquote{pump\_penalty}}}
\pysigstopsignatures
\sphinxAtStartPar
Penalty for adding a pump to the sewer network.
\begin{quote}\begin{description}
\sphinxlineitem{Type}
\sphinxAtStartPar
int

\end{description}\end{quote}

\end{fulllineitems}

\index{add\_node() (pysewer.preprocessing.ModelDomain method)@\spxentry{add\_node()}\spxextra{pysewer.preprocessing.ModelDomain method}}

\begin{fulllineitems}
\phantomsection\label{\detokenize{pysewer:pysewer.preprocessing.ModelDomain.add_node}}
\pysigstartsignatures
\pysiglinewithargsret{\sphinxbfcode{\sphinxupquote{add\_node}}}{\sphinxparam{\DUrole{n}{point}}\sphinxparamcomma \sphinxparam{\DUrole{n}{node\_type}}\sphinxparamcomma \sphinxparam{\DUrole{n}{closest\_edge}\DUrole{o}{=}\DUrole{default_value}{None}}\sphinxparamcomma \sphinxparam{\DUrole{n}{node\_attributes}\DUrole{o}{=}\DUrole{default_value}{None}}}{}
\pysigstopsignatures
\sphinxAtStartPar
Adds a node to the connection graph.
\begin{quote}\begin{description}
\sphinxlineitem{Parameters}\begin{itemize}
\item {} 
\sphinxAtStartPar
\sphinxstyleliteralstrong{\sphinxupquote{point}} (\sphinxstyleliteralemphasis{\sphinxupquote{shapely.geometry.Point}}) \textendash{} The point to add as a node.

\item {} 
\sphinxAtStartPar
\sphinxstyleliteralstrong{\sphinxupquote{node\_type}} (\sphinxstyleliteralemphasis{\sphinxupquote{str}}) \textendash{} The type of node to add.

\item {} 
\sphinxAtStartPar
\sphinxstyleliteralstrong{\sphinxupquote{closest\_edge}} (\sphinxstyleliteralemphasis{\sphinxupquote{shapely.geometry.LineString}}\sphinxstyleliteralemphasis{\sphinxupquote{, }}\sphinxstyleliteralemphasis{\sphinxupquote{optional}}) \textendash{} The closest edge to the point. Defaults to None.

\item {} 
\sphinxAtStartPar
\sphinxstyleliteralstrong{\sphinxupquote{node\_attributes}} (\sphinxstyleliteralemphasis{\sphinxupquote{dict}}\sphinxstyleliteralemphasis{\sphinxupquote{, }}\sphinxstyleliteralemphasis{\sphinxupquote{optional}}) \textendash{} Additional attributes to add to the node. Defaults to None.

\end{itemize}

\sphinxlineitem{Return type}
\sphinxAtStartPar
None

\end{description}\end{quote}
\subsubsection*{Notes}

\sphinxAtStartPar
This method adds a node to the connection graph. If \sphinxtitleref{closest\_edge} is not provided, it finds the closest edge to
the point and uses that as the \sphinxtitleref{closest\_edge}. It then disconnects edges from the node and adds the node to the
connection graph. If there are any cluster centers, it connects the node to the nearest cluster center. If there
are no cluster centers, it connects the node to the road network.

\end{fulllineitems}

\index{add\_sink() (pysewer.preprocessing.ModelDomain method)@\spxentry{add\_sink()}\spxextra{pysewer.preprocessing.ModelDomain method}}

\begin{fulllineitems}
\phantomsection\label{\detokenize{pysewer:pysewer.preprocessing.ModelDomain.add_sink}}
\pysigstartsignatures
\pysiglinewithargsret{\sphinxbfcode{\sphinxupquote{add\_sink}}}{\sphinxparam{\DUrole{n}{sink\_location}\DUrole{p}{:}\DUrole{w}{ }\DUrole{n}{tuple}}}{}
\pysigstopsignatures
\sphinxAtStartPar
Adds a sink node to the graph at the specified location.
\begin{quote}\begin{description}
\sphinxlineitem{Parameters}
\sphinxAtStartPar
\sphinxstyleliteralstrong{\sphinxupquote{sink\_location}} (\sphinxstyleliteralemphasis{\sphinxupquote{tuple}}) \textendash{} A tuple containing the x and y coordinates of the sink location.

\sphinxlineitem{Return type}
\sphinxAtStartPar
None

\end{description}\end{quote}

\end{fulllineitems}

\index{connect\_buildings() (pysewer.preprocessing.ModelDomain method)@\spxentry{connect\_buildings()}\spxextra{pysewer.preprocessing.ModelDomain method}}

\begin{fulllineitems}
\phantomsection\label{\detokenize{pysewer:pysewer.preprocessing.ModelDomain.connect_buildings}}
\pysigstartsignatures
\pysiglinewithargsret{\sphinxbfcode{\sphinxupquote{connect\_buildings}}}{\sphinxparam{\DUrole{n}{max\_connection\_length}\DUrole{p}{:}\DUrole{w}{ }\DUrole{n}{int}\DUrole{w}{ }\DUrole{o}{=}\DUrole{w}{ }\DUrole{default_value}{30}}\sphinxparamcomma \sphinxparam{\DUrole{n}{clustering}\DUrole{p}{:}\DUrole{w}{ }\DUrole{n}{str}\DUrole{w}{ }\DUrole{o}{=}\DUrole{w}{ }\DUrole{default_value}{\textquotesingle{}centers\textquotesingle{}}}}{}
\pysigstopsignatures
\sphinxAtStartPar
Connects the buildings in the network by adding nodes to the graph.
:param max\_connection\_length: The maximum distance between a building and the nearest street for it to be included in the cluster
\begin{quote}

\sphinxAtStartPar
centers list. The default is 30.
\end{quote}
\begin{quote}\begin{description}
\sphinxlineitem{Parameters}
\sphinxAtStartPar
\sphinxstyleliteralstrong{\sphinxupquote{clustering}} (\sphinxstyleliteralemphasis{\sphinxupquote{str}}\sphinxstyleliteralemphasis{\sphinxupquote{, }}\sphinxstyleliteralemphasis{\sphinxupquote{optional}}) \textendash{} The method used to cluster the buildings. Can be “centers” or “none”. The default is “centers”.

\sphinxlineitem{Return type}
\sphinxAtStartPar
None

\end{description}\end{quote}
\subsubsection*{Notes}

\sphinxAtStartPar
This method adds nodes to the graph to connect the buildings in the network. It first gets the building points
and then clusters them based on the specified method. If clustering is set to “centers”, it gets the cluster
centers and finds the closest edges to them. It then adds nodes to the graph for each cluster center, with the
closest edge as an attribute. If clustering is set to “none”, it simply adds nodes to the graph for each building.
In both cases, it finds the closest edges to the buildings and adds nodes to the graph for each building, with
the closest edge as an attribute.
\subsubsection*{Examples}

\begin{sphinxVerbatim}[commandchars=\\\{\}]
\PYG{g+gp}{\PYGZgt{}\PYGZgt{}\PYGZgt{} }\PYG{n}{network} \PYG{o}{=} \PYG{n}{Network}\PYG{p}{(}\PYG{p}{)}
\PYG{g+gp}{\PYGZgt{}\PYGZgt{}\PYGZgt{} }\PYG{n}{network}\PYG{o}{.}\PYG{n}{connect\PYGZus{}buildings}\PYG{p}{(}\PYG{n}{max\PYGZus{}connection\PYGZus{}length}\PYG{o}{=}\PYG{l+m+mi}{50}\PYG{p}{,} \PYG{n}{clustering}\PYG{o}{=}\PYG{l+s+s2}{\PYGZdq{}}\PYG{l+s+s2}{centers}\PYG{l+s+s2}{\PYGZdq{}}\PYG{p}{)}
\end{sphinxVerbatim}

\end{fulllineitems}

\index{connect\_subgraphs() (pysewer.preprocessing.ModelDomain method)@\spxentry{connect\_subgraphs()}\spxextra{pysewer.preprocessing.ModelDomain method}}

\begin{fulllineitems}
\phantomsection\label{\detokenize{pysewer:pysewer.preprocessing.ModelDomain.connect_subgraphs}}
\pysigstartsignatures
\pysiglinewithargsret{\sphinxbfcode{\sphinxupquote{connect\_subgraphs}}}{}{}
\pysigstopsignatures
\sphinxAtStartPar
Identifies unconnect street subnetworks and connects them based on shortest distance

\end{fulllineitems}

\index{connect\_to\_roadnetwork() (pysewer.preprocessing.ModelDomain method)@\spxentry{connect\_to\_roadnetwork()}\spxextra{pysewer.preprocessing.ModelDomain method}}

\begin{fulllineitems}
\phantomsection\label{\detokenize{pysewer:pysewer.preprocessing.ModelDomain.connect_to_roadnetwork}}
\pysigstartsignatures
\pysiglinewithargsret{\sphinxbfcode{\sphinxupquote{connect\_to\_roadnetwork}}}{\sphinxparam{\DUrole{n}{G}\DUrole{p}{:}\DUrole{w}{ }\DUrole{n}{Graph}}\sphinxparamcomma \sphinxparam{\DUrole{n}{new\_node}}\sphinxparamcomma \sphinxparam{\DUrole{n}{conn\_point}}\sphinxparamcomma \sphinxparam{\DUrole{n}{closest\_edge}}\sphinxparamcomma \sphinxparam{\DUrole{n}{add\_private\_sewer}\DUrole{p}{:}\DUrole{w}{ }\DUrole{n}{bool}\DUrole{w}{ }\DUrole{o}{=}\DUrole{w}{ }\DUrole{default_value}{True}}}{}
\pysigstopsignatures
\sphinxAtStartPar
Connects a new node to the road network by inserting a connection point on the closest edge and adjusting edges.
\begin{description}
\sphinxlineitem{Args:}
\sphinxAtStartPar
G (networkx.Graph): The road network graph.
new\_node (NodeTpye): The new node to be connected to the road network.
conn\_point (pysewer.Point): The point where the new node will be connected to the road network.
closest\_edge (pysewer.Edge): The closest edge to the new node.
add\_private\_sewer (bool, optional): Whether to add a private sewer between the new node and the connection point. Defaults to True.

\sphinxlineitem{Returns:}
\sphinxAtStartPar
bool: True if the connection was successful, False otherwise.

\end{description}

\end{fulllineitems}

\index{create\_unsimplified\_graph() (pysewer.preprocessing.ModelDomain method)@\spxentry{create\_unsimplified\_graph()}\spxextra{pysewer.preprocessing.ModelDomain method}}

\begin{fulllineitems}
\phantomsection\label{\detokenize{pysewer:pysewer.preprocessing.ModelDomain.create_unsimplified_graph}}
\pysigstartsignatures
\pysiglinewithargsret{\sphinxbfcode{\sphinxupquote{create\_unsimplified\_graph}}}{\sphinxparam{\DUrole{n}{roads\_gdf}\DUrole{p}{:}\DUrole{w}{ }\DUrole{n}{GeoDataFrame}}}{{ $\rightarrow$ Graph}}
\pysigstopsignatures
\sphinxAtStartPar
Create an unsimplified graph from a GeoDataFrame of roads.
:param roads\_gdf: A GeoDataFrame containing road data.
:type roads\_gdf: gpd.GeoDataFrame
\begin{quote}\begin{description}
\sphinxlineitem{Returns}
\sphinxAtStartPar
An unsimplified graph containing nodes and edges from the GeoDataFrame.

\sphinxlineitem{Return type}
\sphinxAtStartPar
nx.Graph

\end{description}\end{quote}

\end{fulllineitems}

\index{generate\_connection\_graph() (pysewer.preprocessing.ModelDomain method)@\spxentry{generate\_connection\_graph()}\spxextra{pysewer.preprocessing.ModelDomain method}}

\begin{fulllineitems}
\phantomsection\label{\detokenize{pysewer:pysewer.preprocessing.ModelDomain.generate_connection_graph}}
\pysigstartsignatures
\pysiglinewithargsret{\sphinxbfcode{\sphinxupquote{generate\_connection\_graph}}}{}{{ $\rightarrow$ MultiDiGraph}}
\pysigstopsignatures
\sphinxAtStartPar
Generates a connection graph from the given connection data and returns it.
This method simplifies the connection graph, removes any self loops, sets trench depth node attributes to 0,
calculates the geometry, distance, profile, needs\_pump, weight, and elevation attributes for each edge and node
in the connection graph.


\subsection{Returns:}
\label{\detokenize{pysewer:id16}}\begin{description}
\sphinxlineitem{nx.MultiDiGraph}
\sphinxAtStartPar
A directed graph representing the connections between the different points in the network.

\end{description}

\end{fulllineitems}

\index{get\_buildings() (pysewer.preprocessing.ModelDomain method)@\spxentry{get\_buildings()}\spxextra{pysewer.preprocessing.ModelDomain method}}

\begin{fulllineitems}
\phantomsection\label{\detokenize{pysewer:pysewer.preprocessing.ModelDomain.get_buildings}}
\pysigstartsignatures
\pysiglinewithargsret{\sphinxbfcode{\sphinxupquote{get\_buildings}}}{}{}
\pysigstopsignatures
\sphinxAtStartPar
Returns a list of node keys for all buildings in the connection graph.

\end{fulllineitems}

\index{get\_sinks() (pysewer.preprocessing.ModelDomain method)@\spxentry{get\_sinks()}\spxextra{pysewer.preprocessing.ModelDomain method}}

\begin{fulllineitems}
\phantomsection\label{\detokenize{pysewer:pysewer.preprocessing.ModelDomain.get_sinks}}
\pysigstartsignatures
\pysiglinewithargsret{\sphinxbfcode{\sphinxupquote{get\_sinks}}}{}{}
\pysigstopsignatures
\sphinxAtStartPar
Returns a list of node keys for all wastewater treatment plants (wwtp) in the connection graph.

\end{fulllineitems}

\index{reset\_sinks() (pysewer.preprocessing.ModelDomain method)@\spxentry{reset\_sinks()}\spxextra{pysewer.preprocessing.ModelDomain method}}

\begin{fulllineitems}
\phantomsection\label{\detokenize{pysewer:pysewer.preprocessing.ModelDomain.reset_sinks}}
\pysigstartsignatures
\pysiglinewithargsret{\sphinxbfcode{\sphinxupquote{reset\_sinks}}}{}{}
\pysigstopsignatures
\sphinxAtStartPar
Resets the sinks in the connection graph by setting their node\_type attribute to an empty string.
:returns: This method does not return anything.
:rtype: None

\end{fulllineitems}

\index{set\_pump\_penalty() (pysewer.preprocessing.ModelDomain method)@\spxentry{set\_pump\_penalty()}\spxextra{pysewer.preprocessing.ModelDomain method}}

\begin{fulllineitems}
\phantomsection\label{\detokenize{pysewer:pysewer.preprocessing.ModelDomain.set_pump_penalty}}
\pysigstartsignatures
\pysiglinewithargsret{\sphinxbfcode{\sphinxupquote{set\_pump\_penalty}}}{\sphinxparam{\DUrole{n}{pp}}}{}
\pysigstopsignatures
\sphinxAtStartPar
Set the pump penalty for the current instance of the ModelDomain class.
:param pp: The pump penalty to set.
:type pp: float
\begin{quote}\begin{description}
\sphinxlineitem{Return type}
\sphinxAtStartPar
None

\end{description}\end{quote}

\end{fulllineitems}

\index{set\_sink\_lowest() (pysewer.preprocessing.ModelDomain method)@\spxentry{set\_sink\_lowest()}\spxextra{pysewer.preprocessing.ModelDomain method}}

\begin{fulllineitems}
\phantomsection\label{\detokenize{pysewer:pysewer.preprocessing.ModelDomain.set_sink_lowest}}
\pysigstartsignatures
\pysiglinewithargsret{\sphinxbfcode{\sphinxupquote{set\_sink\_lowest}}}{\sphinxparam{\DUrole{n}{candidate\_nodes}\DUrole{p}{:}\DUrole{w}{ }\DUrole{n}{list\DUrole{w}{ }\DUrole{p}{|}\DUrole{w}{ }None}\DUrole{w}{ }\DUrole{o}{=}\DUrole{w}{ }\DUrole{default_value}{None}}}{}
\pysigstopsignatures
\sphinxAtStartPar
Sets the sink node to the lowest point in the graph.
\begin{quote}\begin{description}
\sphinxlineitem{Parameters}
\sphinxAtStartPar
\sphinxstyleliteralstrong{\sphinxupquote{candidate\_nodes}} (\sphinxstyleliteralemphasis{\sphinxupquote{list}}\sphinxstyleliteralemphasis{\sphinxupquote{, }}\sphinxstyleliteralemphasis{\sphinxupquote{optional}}) \textendash{} A list of candidate nodes to consider for the sink node. If None, all nodes except buildings are considered.

\sphinxlineitem{Return type}
\sphinxAtStartPar
None

\end{description}\end{quote}
\subsubsection*{Notes}

\sphinxAtStartPar
This method sets the sink node to the lowest point in the graph. If candidate\_nodes is not None, only the nodes in candidate\_nodes that are not buildings are considered.

\end{fulllineitems}


\end{fulllineitems}

\index{Roads (class in pysewer.preprocessing)@\spxentry{Roads}\spxextra{class in pysewer.preprocessing}}

\begin{fulllineitems}
\phantomsection\label{\detokenize{pysewer:pysewer.preprocessing.Roads}}
\pysigstartsignatures
\pysiglinewithargsret{\sphinxbfcode{\sphinxupquote{class\DUrole{w}{ }}}\sphinxcode{\sphinxupquote{pysewer.preprocessing.}}\sphinxbfcode{\sphinxupquote{Roads}}}{\sphinxparam{\DUrole{n}{input\_data}\DUrole{p}{:}\DUrole{w}{ }\DUrole{n}{str\DUrole{w}{ }\DUrole{p}{|}\DUrole{w}{ }GeoDataFrame}}}{}
\pysigstopsignatures
\sphinxAtStartPar
Bases: \sphinxcode{\sphinxupquote{object}}

\sphinxAtStartPar
A class to represent road data from either a shapefile or a geopandas dataframe.
Attributes:
———\sphinxhyphen{}
gdf : geopandas.GeoDataFrame
\begin{quote}

\sphinxAtStartPar
A geopandas dataframe containing road data.
\end{quote}
\begin{description}
\sphinxlineitem{merged\_roads}{[}shapely.geometry.MultiLineString{]}
\sphinxAtStartPar
A shapely MultiLineString object representing the merged road data.

\end{description}


\subsection{Methods:}
\label{\detokenize{pysewer:methods}}\begin{description}
\sphinxlineitem{\_\_init\_\_(self, input\_data: str or geopandas.GeoDataFrame) \sphinxhyphen{}\textgreater{} None}
\sphinxAtStartPar
Initializes a Roads object with road data from either a shapefile or a geopandas dataframe.

\end{description}
\index{get\_crs() (pysewer.preprocessing.Roads method)@\spxentry{get\_crs()}\spxextra{pysewer.preprocessing.Roads method}}

\begin{fulllineitems}
\phantomsection\label{\detokenize{pysewer:pysewer.preprocessing.Roads.get_crs}}
\pysigstartsignatures
\pysiglinewithargsret{\sphinxbfcode{\sphinxupquote{get\_crs}}}{}{}
\pysigstopsignatures
\sphinxAtStartPar
Gets the coordinate reference system (CRS) of the Roads object.
\begin{quote}\begin{description}
\sphinxlineitem{Returns}
\sphinxAtStartPar
The coordinate system of the Roads object.

\sphinxlineitem{Return type}
\sphinxAtStartPar
dict

\end{description}\end{quote}

\end{fulllineitems}

\index{get\_gdf() (pysewer.preprocessing.Roads method)@\spxentry{get\_gdf()}\spxextra{pysewer.preprocessing.Roads method}}

\begin{fulllineitems}
\phantomsection\label{\detokenize{pysewer:pysewer.preprocessing.Roads.get_gdf}}
\pysigstartsignatures
\pysiglinewithargsret{\sphinxbfcode{\sphinxupquote{get\_gdf}}}{}{}
\pysigstopsignatures
\sphinxAtStartPar
Returns the road data as a geopandas dataframe.
:returns: The road data as a geopandas dataframe.
:rtype: geopandas.GeoDataFrame

\end{fulllineitems}

\index{get\_merged\_roads() (pysewer.preprocessing.Roads method)@\spxentry{get\_merged\_roads()}\spxextra{pysewer.preprocessing.Roads method}}

\begin{fulllineitems}
\phantomsection\label{\detokenize{pysewer:pysewer.preprocessing.Roads.get_merged_roads}}
\pysigstartsignatures
\pysiglinewithargsret{\sphinxbfcode{\sphinxupquote{get\_merged\_roads}}}{}{}
\pysigstopsignatures
\sphinxAtStartPar
Merge the road network as a shapely MultiLineString.
\begin{quote}\begin{description}
\sphinxlineitem{Returns}
\sphinxAtStartPar
merged road network as a shapely MultiLineString

\sphinxlineitem{Return type}
\sphinxAtStartPar
shapely MultiLineString

\end{description}\end{quote}

\end{fulllineitems}

\index{get\_nearest\_point() (pysewer.preprocessing.Roads method)@\spxentry{get\_nearest\_point()}\spxextra{pysewer.preprocessing.Roads method}}

\begin{fulllineitems}
\phantomsection\label{\detokenize{pysewer:pysewer.preprocessing.Roads.get_nearest_point}}
\pysigstartsignatures
\pysiglinewithargsret{\sphinxbfcode{\sphinxupquote{get\_nearest\_point}}}{\sphinxparam{\DUrole{n}{point}}}{}
\pysigstopsignatures
\sphinxAtStartPar
Returns the nearest location to the input point on the street network.
:param point: Point to find nearest location to.
:type point: shapely.geometry.Point
\begin{quote}\begin{description}
\sphinxlineitem{Returns}
\sphinxAtStartPar
Nearest location to the input point on the street network.

\sphinxlineitem{Return type}
\sphinxAtStartPar
shapely.geometry.Point

\end{description}\end{quote}

\end{fulllineitems}


\end{fulllineitems}



\section{Routing}
\label{\detokenize{pysewer:module-pysewer.routing}}\label{\detokenize{pysewer:routing}}\index{module@\spxentry{module}!pysewer.routing@\spxentry{pysewer.routing}}\index{pysewer.routing@\spxentry{pysewer.routing}!module@\spxentry{module}}\index{find\_rsph\_path() (in module pysewer.routing)@\spxentry{find\_rsph\_path()}\spxextra{in module pysewer.routing}}

\begin{fulllineitems}
\phantomsection\label{\detokenize{pysewer:pysewer.routing.find_rsph_path}}
\pysigstartsignatures
\pysiglinewithargsret{\sphinxcode{\sphinxupquote{pysewer.routing.}}\sphinxbfcode{\sphinxupquote{find\_rsph\_path}}}{\sphinxparam{\DUrole{n}{connection\_graph}\DUrole{p}{:}\DUrole{w}{ }\DUrole{n}{Graph}}\sphinxparamcomma \sphinxparam{\DUrole{n}{subgraph\_nodes}\DUrole{p}{:}\DUrole{w}{ }\DUrole{n}{List\DUrole{p}{{[}}int\DUrole{w}{ }\DUrole{p}{|}\DUrole{w}{ }str\DUrole{w}{ }\DUrole{p}{|}\DUrole{w}{ }Hashable\DUrole{p}{{]}}}}\sphinxparamcomma \sphinxparam{\DUrole{n}{terminals}\DUrole{p}{:}\DUrole{w}{ }\DUrole{n}{List\DUrole{p}{{[}}int\DUrole{w}{ }\DUrole{p}{|}\DUrole{w}{ }str\DUrole{w}{ }\DUrole{p}{|}\DUrole{w}{ }Hashable\DUrole{p}{{]}}}}\sphinxparamcomma \sphinxparam{\DUrole{n}{all\_paths}\DUrole{p}{:}\DUrole{w}{ }\DUrole{n}{dict}}\sphinxparamcomma \sphinxparam{\DUrole{n}{all\_lengths}\DUrole{p}{:}\DUrole{w}{ }\DUrole{n}{dict}}}{{ $\rightarrow$ List\DUrole{p}{{[}}int\DUrole{w}{ }\DUrole{p}{|}\DUrole{w}{ }str\DUrole{w}{ }\DUrole{p}{|}\DUrole{w}{ }Hashable\DUrole{p}{{]}}}}
\pysigstopsignatures
\sphinxAtStartPar
Identifies the closest terminal node to the growing tree and returns the connection path.
\begin{quote}\begin{description}
\sphinxlineitem{Parameters}\begin{itemize}
\item {} 
\sphinxAtStartPar
\sphinxstyleliteralstrong{\sphinxupquote{connection\_graph}} (\sphinxstyleliteralemphasis{\sphinxupquote{nx.Graph}}) \textendash{} The graph representing the entire network.

\item {} 
\sphinxAtStartPar
\sphinxstyleliteralstrong{\sphinxupquote{subgraph\_nodes}} (\sphinxstyleliteralemphasis{\sphinxupquote{List}}\sphinxstyleliteralemphasis{\sphinxupquote{{[}}}\sphinxstyleliteralemphasis{\sphinxupquote{NodeType}}\sphinxstyleliteralemphasis{\sphinxupquote{{]}}}) \textendash{} The nodes in the subgraph being grown.

\item {} 
\sphinxAtStartPar
\sphinxstyleliteralstrong{\sphinxupquote{terminals}} (\sphinxstyleliteralemphasis{\sphinxupquote{List}}\sphinxstyleliteralemphasis{\sphinxupquote{{[}}}\sphinxstyleliteralemphasis{\sphinxupquote{NodeType}}\sphinxstyleliteralemphasis{\sphinxupquote{{]}}}) \textendash{} The terminal nodes in the network.

\item {} 
\sphinxAtStartPar
\sphinxstyleliteralstrong{\sphinxupquote{all\_paths}} (\sphinxstyleliteralemphasis{\sphinxupquote{dict}}) \textendash{} A dictionary containing all the shortest paths between terminal nodes.

\item {} 
\sphinxAtStartPar
\sphinxstyleliteralstrong{\sphinxupquote{all\_lengths}} (\sphinxstyleliteralemphasis{\sphinxupquote{dict}}) \textendash{} A dictionary containing the lengths of all the shortest paths between terminal nodes.

\end{itemize}

\sphinxlineitem{Returns}
\sphinxAtStartPar
The shortest path between the closest terminal node and the growing subgraph.

\sphinxlineitem{Return type}
\sphinxAtStartPar
List{[}NodeType{]}

\sphinxlineitem{Raises}
\sphinxAtStartPar
\sphinxstyleliteralstrong{\sphinxupquote{ValueError}} \textendash{} If there is no recorded path from the closest terminal node to the growing subgraph in the all\_paths dictionary.

\end{description}\end{quote}

\end{fulllineitems}

\index{rsph\_tree() (in module pysewer.routing)@\spxentry{rsph\_tree()}\spxextra{in module pysewer.routing}}

\begin{fulllineitems}
\phantomsection\label{\detokenize{pysewer:pysewer.routing.rsph_tree}}
\pysigstartsignatures
\pysiglinewithargsret{\sphinxcode{\sphinxupquote{pysewer.routing.}}\sphinxbfcode{\sphinxupquote{rsph\_tree}}}{\sphinxparam{\DUrole{n}{connection\_graph}\DUrole{p}{:}\DUrole{w}{ }\DUrole{n}{Graph}}\sphinxparamcomma \sphinxparam{\DUrole{n}{sinks}\DUrole{p}{:}\DUrole{w}{ }\DUrole{n}{List\DUrole{p}{{[}}int\DUrole{w}{ }\DUrole{p}{|}\DUrole{w}{ }str\DUrole{w}{ }\DUrole{p}{|}\DUrole{w}{ }Hashable\DUrole{p}{{]}}}}\sphinxparamcomma \sphinxparam{\DUrole{n}{from\_type}\DUrole{p}{:}\DUrole{w}{ }\DUrole{n}{str}\DUrole{w}{ }\DUrole{o}{=}\DUrole{w}{ }\DUrole{default_value}{\textquotesingle{}building\textquotesingle{}}}\sphinxparamcomma \sphinxparam{\DUrole{n}{to\_type}\DUrole{p}{:}\DUrole{w}{ }\DUrole{n}{str}\DUrole{w}{ }\DUrole{o}{=}\DUrole{w}{ }\DUrole{default_value}{\textquotesingle{}wwtp\textquotesingle{}}}\sphinxparamcomma \sphinxparam{\DUrole{n}{skip\_nodes}\DUrole{p}{:}\DUrole{w}{ }\DUrole{n}{List\DUrole{p}{{[}}int\DUrole{w}{ }\DUrole{p}{|}\DUrole{w}{ }str\DUrole{w}{ }\DUrole{p}{|}\DUrole{w}{ }Hashable\DUrole{p}{{]}}}\DUrole{w}{ }\DUrole{o}{=}\DUrole{w}{ }\DUrole{default_value}{{[}{]}}}}{{ $\rightarrow$ DiGraph}}
\pysigstopsignatures
\sphinxAtStartPar
Returns the directed routed steiner tree that connects all terminal nodes to the sink using the repeated shortest path heuristic.
\begin{quote}\begin{description}
\sphinxlineitem{Parameters}\begin{itemize}
\item {} 
\sphinxAtStartPar
\sphinxstyleliteralstrong{\sphinxupquote{connection\_graph}} (\sphinxstyleliteralemphasis{\sphinxupquote{nx.Graph}}) \textendash{} The graph representing the sewer network.

\item {} 
\sphinxAtStartPar
\sphinxstyleliteralstrong{\sphinxupquote{sinks}} (\sphinxstyleliteralemphasis{\sphinxupquote{List}}\sphinxstyleliteralemphasis{\sphinxupquote{{[}}}\sphinxstyleliteralemphasis{\sphinxupquote{NodeType}}\sphinxstyleliteralemphasis{\sphinxupquote{{]}}}) \textendash{} The nodes representing the sinks in the sewer network.

\item {} 
\sphinxAtStartPar
\sphinxstyleliteralstrong{\sphinxupquote{from\_type}} (\sphinxstyleliteralemphasis{\sphinxupquote{str}}\sphinxstyleliteralemphasis{\sphinxupquote{, }}\sphinxstyleliteralemphasis{\sphinxupquote{optional}}) \textendash{} The type of the source nodes, by default “building”.

\item {} 
\sphinxAtStartPar
\sphinxstyleliteralstrong{\sphinxupquote{to\_type}} (\sphinxstyleliteralemphasis{\sphinxupquote{str}}\sphinxstyleliteralemphasis{\sphinxupquote{, }}\sphinxstyleliteralemphasis{\sphinxupquote{optional}}) \textendash{} The type of the sink nodes, by default “wwtp”.

\item {} 
\sphinxAtStartPar
\sphinxstyleliteralstrong{\sphinxupquote{skip\_nodes}} (\sphinxstyleliteralemphasis{\sphinxupquote{List}}\sphinxstyleliteralemphasis{\sphinxupquote{{[}}}\sphinxstyleliteralemphasis{\sphinxupquote{NodeType}}\sphinxstyleliteralemphasis{\sphinxupquote{{]}}}\sphinxstyleliteralemphasis{\sphinxupquote{, }}\sphinxstyleliteralemphasis{\sphinxupquote{optional}}) \textendash{} The nodes to skip in the routing, by default {[}{]}.

\end{itemize}

\sphinxlineitem{Returns}
\sphinxAtStartPar
The directed routed steiner tree that connects all terminal nodes to the sink using the repeated shortest path heuristic.

\sphinxlineitem{Return type}
\sphinxAtStartPar
nx.DiGraph

\end{description}\end{quote}

\end{fulllineitems}



\section{Simplify Graph}
\label{\detokenize{pysewer:module-pysewer.simplify}}\label{\detokenize{pysewer:simplify-graph}}\index{module@\spxentry{module}!pysewer.simplify@\spxentry{pysewer.simplify}}\index{pysewer.simplify@\spxentry{pysewer.simplify}!module@\spxentry{module}}\index{get\_essential\_nodes() (in module pysewer.simplify)@\spxentry{get\_essential\_nodes()}\spxextra{in module pysewer.simplify}}

\begin{fulllineitems}
\phantomsection\label{\detokenize{pysewer:pysewer.simplify.get_essential_nodes}}
\pysigstartsignatures
\pysiglinewithargsret{\sphinxcode{\sphinxupquote{pysewer.simplify.}}\sphinxbfcode{\sphinxupquote{get\_essential\_nodes}}}{\sphinxparam{\DUrole{n}{G}}}{}
\pysigstopsignatures
\sphinxAtStartPar
Returns a list of essential nodes to build the simplified graph. This includes all junctions (degree \textgreater{} 2), buildings, and connection points.
\begin{quote}\begin{description}
\sphinxlineitem{Parameters}
\sphinxAtStartPar
\sphinxstyleliteralstrong{\sphinxupquote{G}} (\sphinxstyleliteralemphasis{\sphinxupquote{networkx.Graph}}) \textendash{} NetworkX street graph with connected buildings.

\sphinxlineitem{Returns}
\sphinxAtStartPar
List of node keys.

\sphinxlineitem{Return type}
\sphinxAtStartPar
List

\end{description}\end{quote}
\subsubsection*{Examples}

\begin{sphinxVerbatim}[commandchars=\\\{\}]
\PYG{g+gp}{\PYGZgt{}\PYGZgt{}\PYGZgt{} }\PYG{n}{G} \PYG{o}{=} \PYG{n}{nx}\PYG{o}{.}\PYG{n}{Graph}\PYG{p}{(}\PYG{p}{)}
\PYG{g+gp}{\PYGZgt{}\PYGZgt{}\PYGZgt{} }\PYG{n}{G}\PYG{o}{.}\PYG{n}{add\PYGZus{}node}\PYG{p}{(}\PYG{l+m+mi}{1}\PYG{p}{,} \PYG{n}{road\PYGZus{}network}\PYG{o}{=}\PYG{k+kc}{True}\PYG{p}{)}
\PYG{g+gp}{\PYGZgt{}\PYGZgt{}\PYGZgt{} }\PYG{n}{G}\PYG{o}{.}\PYG{n}{add\PYGZus{}node}\PYG{p}{(}\PYG{l+m+mi}{2}\PYG{p}{,} \PYG{n}{road\PYGZus{}network}\PYG{o}{=}\PYG{k+kc}{True}\PYG{p}{)}
\PYG{g+gp}{\PYGZgt{}\PYGZgt{}\PYGZgt{} }\PYG{n}{G}\PYG{o}{.}\PYG{n}{add\PYGZus{}node}\PYG{p}{(}\PYG{l+m+mi}{3}\PYG{p}{,} \PYG{n}{road\PYGZus{}network}\PYG{o}{=}\PYG{k+kc}{False}\PYG{p}{)}
\PYG{g+gp}{\PYGZgt{}\PYGZgt{}\PYGZgt{} }\PYG{n}{G}\PYG{o}{.}\PYG{n}{add\PYGZus{}node}\PYG{p}{(}\PYG{l+m+mi}{4}\PYG{p}{,} \PYG{n}{connection\PYGZus{}node}\PYG{o}{=}\PYG{k+kc}{True}\PYG{p}{)}
\PYG{g+gp}{\PYGZgt{}\PYGZgt{}\PYGZgt{} }\PYG{n}{G}\PYG{o}{.}\PYG{n}{add\PYGZus{}node}\PYG{p}{(}\PYG{l+m+mi}{5}\PYG{p}{,} \PYG{n}{connection\PYGZus{}node}\PYG{o}{=}\PYG{k+kc}{False}\PYG{p}{)}
\PYG{g+gp}{\PYGZgt{}\PYGZgt{}\PYGZgt{} }\PYG{n}{G}\PYG{o}{.}\PYG{n}{add\PYGZus{}edge}\PYG{p}{(}\PYG{l+m+mi}{1}\PYG{p}{,} \PYG{l+m+mi}{2}\PYG{p}{)}
\PYG{g+gp}{\PYGZgt{}\PYGZgt{}\PYGZgt{} }\PYG{n}{G}\PYG{o}{.}\PYG{n}{add\PYGZus{}edge}\PYG{p}{(}\PYG{l+m+mi}{2}\PYG{p}{,} \PYG{l+m+mi}{3}\PYG{p}{)}
\PYG{g+gp}{\PYGZgt{}\PYGZgt{}\PYGZgt{} }\PYG{n}{G}\PYG{o}{.}\PYG{n}{add\PYGZus{}edge}\PYG{p}{(}\PYG{l+m+mi}{2}\PYG{p}{,} \PYG{l+m+mi}{4}\PYG{p}{)}
\PYG{g+gp}{\PYGZgt{}\PYGZgt{}\PYGZgt{} }\PYG{n}{G}\PYG{o}{.}\PYG{n}{add\PYGZus{}edge}\PYG{p}{(}\PYG{l+m+mi}{4}\PYG{p}{,} \PYG{l+m+mi}{5}\PYG{p}{)}
\PYG{g+gp}{\PYGZgt{}\PYGZgt{}\PYGZgt{} }\PYG{n}{get\PYGZus{}essential\PYGZus{}nodes}\PYG{p}{(}\PYG{n}{G}\PYG{p}{)}
\PYG{g+go}{[1, 2, 3, 5]}
\end{sphinxVerbatim}

\end{fulllineitems}

\index{simplify\_graph() (in module pysewer.simplify)@\spxentry{simplify\_graph()}\spxextra{in module pysewer.simplify}}

\begin{fulllineitems}
\phantomsection\label{\detokenize{pysewer:pysewer.simplify.simplify_graph}}
\pysigstartsignatures
\pysiglinewithargsret{\sphinxcode{\sphinxupquote{pysewer.simplify.}}\sphinxbfcode{\sphinxupquote{simplify\_graph}}}{\sphinxparam{\DUrole{n}{G}\DUrole{p}{:}\DUrole{w}{ }\DUrole{n}{MultiDiGraph}}\sphinxparamcomma \sphinxparam{\DUrole{n}{strict}\DUrole{p}{:}\DUrole{w}{ }\DUrole{n}{bool}\DUrole{w}{ }\DUrole{o}{=}\DUrole{w}{ }\DUrole{default_value}{True}}\sphinxparamcomma \sphinxparam{\DUrole{n}{remove\_rings}\DUrole{p}{:}\DUrole{w}{ }\DUrole{n}{bool}\DUrole{w}{ }\DUrole{o}{=}\DUrole{w}{ }\DUrole{default_value}{True}}}{{ $\rightarrow$ MultiDiGraph}}
\pysigstopsignatures
\sphinxAtStartPar
Simplify a graph’s topology by removing interstitial nodes.
Simplify graph topology by removing all nodes that are not intersections
or dead\sphinxhyphen{}ends. Create an edge directly between the end points that
encapsulate them, but retain the geometry of the original edges, saved as
an attribute in new edge. Some of the resulting consolidated edges may
comprise multiple OSM ways, and if so, their multiple attribute values are
stored as a list.
:param G: input graph
:type G: networkx.MultiDiGraph
:param strict: if False, allow nodes to be end points even if they fail all other
\begin{quote}

\sphinxAtStartPar
rules but have incident edges with different OSM IDs. Lets you keep
nodes at elbow two\sphinxhyphen{}way intersections, but sometimes individual blocks
have multiple OSM IDs within them too.
\end{quote}
\begin{quote}\begin{description}
\sphinxlineitem{Parameters}
\sphinxAtStartPar
\sphinxstyleliteralstrong{\sphinxupquote{remove\_rings}} (\sphinxstyleliteralemphasis{\sphinxupquote{bool}}) \textendash{} if True, remove isolated self\sphinxhyphen{}contained rings that have no endpoints

\sphinxlineitem{Returns}
\sphinxAtStartPar
\sphinxstylestrong{G} \textendash{} topologically simplified graph

\sphinxlineitem{Return type}
\sphinxAtStartPar
networkx.MultiDiGraph

\end{description}\end{quote}

\end{fulllineitems}



\chapter{Indices and tables}
\label{\detokenize{index:indices-and-tables}}\begin{itemize}
\item {} 
\sphinxAtStartPar
\DUrole{xref,std,std-ref}{genindex}

\item {} 
\sphinxAtStartPar
\DUrole{xref,std,std-ref}{modindex}

\item {} 
\sphinxAtStartPar
\DUrole{xref,std,std-ref}{search}

\end{itemize}


\renewcommand{\indexname}{Python Module Index}
\begin{sphinxtheindex}
\let\bigletter\sphinxstyleindexlettergroup
\bigletter{p}
\item\relax\sphinxstyleindexentry{pysewer.export}\sphinxstyleindexpageref{pysewer:\detokenize{module-pysewer.export}}
\item\relax\sphinxstyleindexentry{pysewer.helper}\sphinxstyleindexpageref{pysewer:\detokenize{module-pysewer.helper}}
\item\relax\sphinxstyleindexentry{pysewer.optimization}\sphinxstyleindexpageref{pysewer:\detokenize{module-pysewer.optimization}}
\item\relax\sphinxstyleindexentry{pysewer.plotting}\sphinxstyleindexpageref{pysewer:\detokenize{module-pysewer.plotting}}
\item\relax\sphinxstyleindexentry{pysewer.preprocessing}\sphinxstyleindexpageref{pysewer:\detokenize{module-pysewer.preprocessing}}
\item\relax\sphinxstyleindexentry{pysewer.routing}\sphinxstyleindexpageref{pysewer:\detokenize{module-pysewer.routing}}
\item\relax\sphinxstyleindexentry{pysewer.simplify}\sphinxstyleindexpageref{pysewer:\detokenize{module-pysewer.simplify}}
\end{sphinxtheindex}

\renewcommand{\indexname}{Index}
\printindex
\end{document}